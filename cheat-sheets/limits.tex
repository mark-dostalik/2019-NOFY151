\documentclass[columns=2]{cheatsheet}

% encoding, language
\usepackage[T1]{fontenc}
\usepackage[czech]{babel}
\usepackage[utf8]{inputenc}

% custom macros
\usepackage{../macros}

% title
\title{Počítání limit funkcí}
\author{Mark Dostal\'{i}k \\ \href{mailto:mark.dostalik@gmail.com}{mark.dostalik@gmail.com}}
\date{\today}

\begin{document}
\maketitle
\subsection{Binomická věta}
\begin{equation*}
  \parens{x + y}^n = \sum_{k=0}^{n} \binom{n}{k} x^{n - k} y^k.
\end{equation*}
Speciálně, pro $n = 2, 3$
\begin{align*}
  \parens{x + y}^2 
  &= 
  x^2 + 2xy + y^2,
  \\
  \parens{x - y}^2 
  &= 
  x^2 - 2xy + y^2,
  \\  
  \parens{x + y}^3 
  &= 
  x^3 + 3 x^2 y + 3 x y^2 + y^3,
  \\
  \parens{x - y}^3 
  &= 
  x^3 - 3 x^2 y + 3 x y^2 - y^3.
\end{align*}

\subsection{Faktorizace $x^n - y^n$}
\begin{equation*}
  x^n - y^n = \parens{x - y} \parens{\sum_{k=0}^{n - 1} x^{n -1 -k} y^k}.
\end{equation*}
Speciálně, pro $n = 2, 3$
\begin{align*}
  x^2 - y^2 
  &= 
  \parens{x - y} \parens{x + y},
  \\
  x^3 - y^3 
  &= 
  \parens{x - y} \parens{x^2 + xy + y^2}.
\end{align*}

\vfill\null\columnbreak

\subsection{Aritmetika limit}
Pro $x_0 \in \R^*$ platí
\begin{align*}
  \lim_{x \to x_0} \brackets{f(x) + g(x)}
  &=
  \lim_{x \to x_0} f(x) + \lim_{x \to x_0} g(x),
  \\
  \lim_{x \to x_0} f(x) g(x)
  &=
  \lim_{x \to x_0} f(x) \lim_{x \to x_0} g(x),
  \\
  \lim_{x \to x_0} \frac{f(x)}{g(x)}
  &=
  \frac{\lim_{x \to x_0} f(x)}{\lim_{x \to x_0} g(x)},
\end{align*}
pokud pravé strany mají smysl v $\R^*$. Výrazy
\begin{equation*}
  \pm \parens{+ \infty - \infty}, 
  \qquad 
  \frac{\pm \infty}{\pm \infty},
  \qquad
  \pm \infty \cdot 0,
  \qquad
  \frac{a}{0}.
\end{equation*}
kde $a \in \R^*$, nemají smysl v $\R^*$.

%\subsection{Limita racionální funkce}
%\begin{equation*}
%  \lim_{x \to \pm \infty} \frac{\sum_{i=0}^{n} a_i x^i}{\sum_{j=0}^{m} b_j x^j}
%  =
%  \begin{cases}
%    1,
%    \\
%    2.
%  \end{cases}
%\end{equation*}

\subsection{Důležité limity}
\setlength{\jot}{6pt}
\begin{align*}
  \lim_{x \to 0} \frac{\sqrt[n]{1 + x} - 1}{x} 
  &=
  \frac{1}{n},
  \\
  \lim_{x \to 0} \frac{\sin x}{x} 
  &= 
  1,
  \\
  \lim_{x \to 0} \frac{1 - \cos x}{x^2} 
  &= 
  \frac{1}{2},
  \\
  \lim_{x \to 0} \frac{\exponential{x} - 1}{x}
  &=
  1,
  \\
  \lim_{x \to 0} \frac{\ln\parens{1 + x}}{x}
  &=
  1.
\end{align*}

\subsection{Limita funkce tvaru $f(x)^{g(x)}$}
\begin{equation*}
  \lim_{x \to x_0} \brackets{f(x)}^{g(x)}
  =
  \exponential{\lim_{x \to x_0} g(x) \ln \brackets{f(x)}}
\end{equation*}

%\subsection{L'Hospitalovo pravidlo}
%\begin{equation*}
%  \lim_{x \to x_0} \frac{f(x)}{g(x)}
%  =
%  \lim_{x \to x_0} \frac{f'(x)}{g'(x)},
%\end{equation*}
%pokud nastává jedna z následujících možností
%\begin{enumerate}
%  \item $\lim_{x \to x_0} f(x) = \lim_{x \to x_0} g(x) = 0$,
%  \item $\lim_{x \to x_0} g(x) = \pm \infty$.
%\end{enumerate}
%
%\subsection{Taylorův rozvoj}

\end{document}
