\documentclass[columns=2]{cheatsheet}

% encoding, language
\usepackage[T1]{fontenc}
\usepackage[czech]{babel}
\usepackage[utf8]{inputenc}

% custom macros
\usepackage{../macros}

% title
\title{Goniometrické vzorce}
\author{Mark Dostal\'{i}k \\ \href{mailto:mark.dostalik@gmail.com}{mark.dostalik@gmail.com}}
\date{\today}

\begin{document}
\maketitle
\subsection{Sudost/lichost}
\begin{align*}
  \sin x 
  &= 
  - \sin \parens{-x},
  \\
  \cos x
  &=
  \cos \parens{-x},
  \\
  \tan x
  &=
  - \tan \parens{-x},
  \\
  \cot x
  &=
  - \cot \parens{-x}.
\end{align*}

\subsection{Posunutí argumentu}
\begin{align*}
  \sin x 
  &= 
  \cos \parens{\frac{\pi}{2} - x},
  \\
  \cos x
  &=
  \sin \parens{\frac{\pi}{2} - x},
  \\
  \tan x
  &=
  \cot \parens{\frac{\pi}{2} - x},
  \\
  \cot x
  &=
  \tan \parens{\frac{\pi}{2} - x}.
\end{align*}

\subsection{Součtové vzorce}
\setlength{\jot}{4.25pt}  % increase vertical spacing between equations
\setlength{\belowdisplayskip}{9pt}  % increase vertical skip after align environment
\begin{align*}
  \sin \parens{x \pm y} 
  &= 
  \sin x \cos y \pm \cos x \sin y,
  \\
  \cos \parens{x \pm y} 
  &=
  \cos x \cos y \mp \sin x \sin y,
  \\
  \tan \parens{x \pm y} 
  &=
  \frac{\tan x \pm \tan y}{1 \mp \tan x \tan y},
  \\
  \cot \parens{x \pm y} 
  &=
  \frac{\cot x \cot y \mp 1}{\cot x \pm \cot y}.
\end{align*}
Speciálně, pro \emph{dvojnásobný úhel} platí
\begin{align*}
  \sin \parens{2x}
  &= 
  2 \sin x \cos x,
  \\
  \cos \parens{2x}
  &= 
  \cos^2 x - \sin^2 x,
  \\
  \tan \parens{2x}
  &= 
  \frac{2 \tan x}{1 - \tan^2 x},
  \\
  \cot \parens{2x}
  &=
  \frac{\cot^2 x - 1}{2 \cot x}.
\end{align*}
Pro \emph{poloviční úhel} pak odsud dále dostáváme
\begin{align*}
  \abs{\sin \parens{\frac{x}{2}}}
  &= 
  \sqrt{\frac{1 - \cos x}{2}},
  \\
  \abs{\cos \parens{\frac{x}{2}}}
  &= 
  \sqrt{\frac{1 + \cos x}{2}},
  \\
  \abs{\tan \parens{\frac{x}{2}}}
  &= 
  \sqrt{\frac{1 - \cos x}{1 + \cos x}},
  \\
  \abs{\cot \parens{\frac{x}{2}}}
  &= 
  \sqrt{\frac{1 + \cos x}{1 - \cos x}}.
\end{align*}

\vfill\null\columnbreak

\subsection{Součty a rozdíly goniometrických funkcí}
\setlength{\abovedisplayskip}{6pt}  % increase vertical skip after align environment
\begin{align*}
  \sin x \pm \sin y 
  &= 
  2 \sin \parens{\frac{x \pm y}{2}} \cos \parens{\frac{x \mp y}{2}},
  \\
  \cos x + \cos y 
  &= 
  2 \cos \parens{\frac{x + y}{2}} \cos \parens{\frac{x - y}{2}},
  \\
  \cos x - \cos y 
  &= 
  - 2 \sin \parens{\frac{x + y}{2}} \sin \parens{\frac{x - y}{2}},
  \\
  \tan x \pm \tan y 
  &=
  \frac{\sin \parens{x \pm y}}{\cos x \cos y},
  \\
  \cot x \pm \cot y
  &=
  \frac{\sin \parens{y \pm x}}{\sin x \sin y},
  \\
  \tan x \pm \cot y
  &=
  \pm \frac{\cos \parens{x \mp y}}{\cos x \sin y}.
\end{align*}

\subsection{Součiny goniometrických funkcí}
\begin{align*}
  \sin x \sin y 
  &= 
  \frac{1}{2} \brackets{\cos \parens{x - y} - \cos \parens{x + y}},
  \\
  \cos x \cos y 
  &= 
  \frac{1}{2} \brackets{\cos \parens{x - y} + \cos \parens{x + y}},
  \\
  \sin x \cos y 
  &= 
  \frac{1}{2} \brackets{\sin \parens{x - y} + \sin \parens{x + y}},
  \\
  \tan x \tan y 
  &=
  \frac{\tan x + \tan y}{\cot x + \cot y},
  \\
  \cot x \cot y
  &=
  \frac{\cot x + \cot y}{\tan x + \tan y},
  \\
  \tan x \cot y
  &=
  \frac{\tan x + \cot y}{\cot x + \tan y}.
\end{align*}
Speciálně, pro \emph{mocniny goniometrických funkcí} máme
\begin{align*}
  \sin^2 x
  &= 
  \frac{1}{2} \brackets{1 - \cos \parens{2x}},
  \\
  \cos^2 x
  &= 
  \frac{1}{2} \brackets{1 + \cos \parens{2x}},
  \\
  \sin^3 x
  &= 
  \frac{1}{4} \brackets{3 \sin x - \sin \parens{3x}},
  \\
  \cos^3 x
  &= 
  \frac{1}{4} \brackets{3 \cos x + \cos \parens{3x}}.
\end{align*}

\end{document}
