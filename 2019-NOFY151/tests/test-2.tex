\documentclass[answers]{exam}

% encoding, language
\usepackage[T1]{fontenc}
\usepackage[czech]{babel}
\usepackage[utf8]{inputenc}

% geometry
\usepackage[a4paper]{geometry}
\geometry{
  a4paper,
  total={170mm,257mm},
  left=20mm,
  top=20mm}

% mathematics
\usepackage{amsmath}
\usepackage{amssymb}
\usepackage{amsthm}

% text appearence
\usepackage{libertine}
\usepackage{microtype}  % better general appearence of text

% graphics
\usepackage{graphicx}
\graphicspath{{figures/}}
\usepackage{tikz}
\usetikzlibrary{calc}
\usepackage{xcolor}

% colors
\definecolor{LightBlue}{HTML}{42bbed}
\definecolor{LightGray}{HTML}{616a6b}

% hypertext
\usepackage{hyperref}
\hypersetup{
	colorlinks=true,
	linkcolor=black,
	urlcolor=LightBlue
}

% bibliography
\usepackage{natbib}

% miscellaneous
\usepackage[parfill]{parskip}
\usepackage{nopageno}
\pagestyle{plain}
\usepackage{titling}

% exam class parameters
\bracketedpoints
\pointpoints{b}{b}
\renewcommand{\solutiontitle}{\noindent\textbf{Řešení: }}

% custom macros
\usepackage{../../macros}

% title
\title{\vspace{-3ex}Matematická analýza I (NOFY151) – 2. zápočtový test\vspace{-1ex}}
\author{\vspace{-2ex}}
\date{\vspace{-2ex}3. prosince 2019}

\begin{document}
\maketitle

\begin{questions}
  \question[2] Určete definiční obor funkce
  \begin{equation*}
    f(x) \defeq \parens{\ln x}^x,
  \end{equation*}
  a v jeho vnitřních bodech spočtěte derivaci $f$.
  
  \begin{solution}
    Přepišme si funkci do tvaru
    \begin{equation*}
      f(x) = \exponential{x \ln \parens{\ln x}}.
    \end{equation*}
    Odsud vidíme, že funkce je definovaná pro $x > 1$, neboť \uv{vnější} přirozený logaritmus musí brát kladné hodnoty a \uv{vnitřní} přirozený logaritmus je kladný pro $x > 1$. Tedy $D_f = \parens{1, +\infty}$.
    
    Derivaci ve všech bodech definičního oboru (všechny jsou vnitřní) pak spočítáme pomocí vět o derivaci složené funkce a derivace součinu dvou funkcí
    \begin{align*}
      \dd{f}{x}(x) 
      &= 
      \dd{}{x} \parens{\exponential{x \ln \parens{\ln x}}} 
      =
      \exponential{x \ln \parens{\ln x}} \dd{}{x} \parens{x \ln \parens{\ln x}}
      \\
      &=
      \exponential{x \ln \parens{\ln x}} \parens{\ln \parens{\ln x} + x \frac{1}{\ln x} \cdot \frac{1}{x}}
      =
      \parens{\ln x}^x \parens{\ln \parens{\ln x} + \frac{1}{\ln x}}.
    \end{align*}
  \end{solution}
  
  \question[4] Spočtěte
  \begin{equation*}
    \int \frac{1}{\cos^3 x} \diff x,
  \end{equation*}
  na maximálních možných intervalech (a ty určete).
  
  \begin{solution}
    Integrand je definovaný a spojitý na intervalech, kde je kosinus vždy kladný nebo vždy záporný, tj. $\parens{(2k - 1)\pi/2, (2k + 1)\pi/2}$. Na těchto intervalech bude mít tedy primitivní funkci, kterou nyní najdeme. 
    
    Nejprve integrand rozšíříme funkcí $\cos x$, využijeme identity $\sin^2 x + \cos^2 x = 1$, a pomocí první substituční metody ($u = \sin x$) dostaneme
    \begin{equation*}
      \int \frac{1}{\cos^3 x} \diff x
      =
      \int \frac{\cos x}{(1 - \sin^2 x)^2} \diff x
      =
      \left| 
        \begin{aligned}
          u &= \sin x
          \\
          \diff u &= \cos x \, \diff x
        \end{aligned}
      \right|
      =
      \int \frac{1}{\parens{1-u^2}^2} \diff u
      =
      \int \frac{1}{\parens{1-u}^2 \parens{1+u}^2} \diff u.    
    \end{equation*}
    Integrand posledního integrálu můžeme rozložit na parciální zlomky
    \begin{equation*}
      \int \frac{1}{\parens{1-u}^2 \parens{1+u}^2} \diff u
      =
      \int
        \left(
          \frac{A}{1 - u} + \frac{B}{(1 - u)^2} + \frac{C}{1 + u} + \frac{D}{(1 + u)^2}
        \right)
      \diff u,
    \end{equation*}
    kde pro reálné konstanty $A, B, C, D$ musí platit
    \begin{equation*}
      1 = A (1 - u)(1 + u)^2 + B (1 + u)^2 + C (1 - u)^2 (1 + u) + D (1 - u)^2.
    \end{equation*}
    Dosazením $u = 1$ dostaneme $B = 1/4$, dosazením $u = -1$ potom $D = 1/4$. Srovnání absolutních členů ($u^0$) dává
    \begin{equation*}
      1 = A + B + C + D.
    \end{equation*}
    Pro koeficienty členů $u^3$ dále musí platit
    \begin{equation*}
      0 = -A + C.
    \end{equation*}
    Z posledních dvou rovnic pak už po dosazení za $B$ a $D$ dostáváme $A = 1/4$, $C = 1/4$.
    
    Rozklad na parciální zlomky nám tedy dává
    \begin{align*}
      \int \frac{1}{\parens{1-u}^2 \parens{1+u}^2} \diff u
      &=
      \frac{1}{4}
      \int
        \left(
          \frac{1}{1 - u} + \frac{1}{(1 - u)^2} + \frac{1}{1 + u} + \frac{1}{(1 + u)^2}
        \right)
      \diff u
      \\
      &=
      \frac{1}{4}
      \parens{
        - \ln \abs{1 - u} + \frac{1}{1 - u} + \ln \abs{1 + u} - \frac{1}{1 + u}
      }
      +
      c,
    \end{align*}
    a pro původní integrál tak po dosazení za $u$ dostáváme
    \begin{equation*}
      \int \frac{1}{\cos^3 x} \diff x
      =
      \frac{1}{4}
      \parens{
        - \ln \abs{1 - \sin x} + \frac{1}{1 - \sin x} + \ln \abs{1 + \sin x} - \frac{1}{1 + \sin x}
      }
      +
      c.
    \end{equation*}
    Absolutní hodnoty nejsou nutné, $\sin x$ na příslušných intervalech nabývá hodnot (-1,1). Výsledek se dá navíc ještě dále upravit do tvaru
    \begin{equation*}
      \int \frac{1}{\cos^3 x} \diff x
      =
      \frac{1}{4}
      \parens{
        \frac{2 \sin x}{\cos^2 x} + \ln \parens{\frac{1 + \sin x}{1 - \sin x}}
      }
      +
      c.
    \end{equation*}    
  \end{solution}
  
  \question[2] Spočtěte limitu funkce
  \begin{equation*}
    \lim_{x \to 0} \parens{\frac{1}{x} - \frac{1}{\exponential{x} - 1}}.
  \end{equation*}
  
  \begin{solution}
  Členy v závorce převedeme na společného jmenovatele a počítáme
  \begin{equation*}
    \lim_{x \to 0} \parens{\frac{1}{x} - \frac{1}{\exponential{x} - 1}}
    =
    \lim_{x \to 0} \frac{\exponential{x} - 1 - x}{x \parens{\exponential{x} - 1}}
    \texteq{l'H}
    \lim_{x \to 0} \frac{\exponential{x} - 1}{\exponential{x} - 1 + x \exponential{x}}
    \texteq{l'H}
    \lim_{x \to 1} \frac{\exponential{x}}{\exponential{x} + \exponential{x} + x\exponential{x}}
    =
    \frac{1}{2},
  \end{equation*}
  kde jsme v prvních dvou rovnostech využili l'Hôpitalova pravidla pro výpočet limity typu ``$\frac{0}{0}$''.
  
  Výraz za druhou rovností lze alternativně před aplikací l'Hôpitalova pravidla upravit do tvaru
  \begin{equation*}
  \lim_{x \to 0} \frac{\exponential{x} - 1 - x}{x \parens{\exponential{x} - 1}}
  =
  \lim_{x \to 0} \frac{x}{\exponential{x} - 1} \frac{\exponential{x} - 1 - x}{x^2}
  =
  \lim_{x \to 0} \frac{\exponential{x} - 1 - x}{x^2},
  \end{equation*}
  kde jsme využili aritmetiku limit a znalost limity
  \begin{equation*}
    \lim_{x \to 0} \frac{\exponential{x} - 1}{x} = 1.
  \end{equation*}
  \end{solution}
  
  \question[2] Spočtěte limitu posloupnosti
  \begin{equation*}
    \lim_{n \to +\infty} n \parens{\sqrt[n]{7} - 1}.
  \end{equation*}
  
  \begin{solution}
  Vypočítejme nejprve limitu funkce
  \begin{equation*}
    \lim_{x \to 0} \frac{7^x - 1}{x}
    \texteq{l'H}
    \lim_{x \to 0} \frac{7^x \ln 7}{1}
    =
    \ln 7,
  \end{equation*}
  kde jsme v první rovnosti využili l'Hôpitalova pravidla pro výpočet limity typu ``$\frac{0}{0}$''. Použijeme-li nyní Heineho větu s volbou $x_n \defeq \frac{1}{n}$, dostáváme, že limita zadané posloupnosti je taktéž rovna $\ln 7$.  
  
  Limita výše se dá samozřejmě spočítat i bez použití l'Hôpitalova pravidla následovně
  \begin{equation*}
    \lim_{x \to 0} \frac{7^x - 1}{x}
    =
    \lim_{x \to 0} \frac{\exponential{x \ln 7} - 1}{x \ln 7} \ln 7
    =
    \ln 7,
  \end{equation*}
  kde jsme využili větu o limitě složené funkce a znalost limity
  \begin{equation*}
    \lim_{x \to 0} \frac{\exponential{x} - 1}{x} = 1.
  \end{equation*}
  \end{solution}
   
\end{questions}

\end{document}
