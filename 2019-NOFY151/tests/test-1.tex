\documentclass[answers]{exam}

% encoding, language
\usepackage[T1]{fontenc}
\usepackage[czech]{babel}
\usepackage[utf8]{inputenc}

% geometry
\usepackage[a4paper]{geometry}
\geometry{
  a4paper,
  total={170mm,257mm},
  left=20mm,
  top=20mm}

% mathematics
\usepackage{amsmath}
\usepackage{amssymb}
\usepackage{amsthm}

% text appearence
\usepackage{libertine}
\usepackage{microtype}  % better general appearence of text

% graphics
\usepackage{graphicx}
\graphicspath{{figures/}}
\usepackage{tikz}
\usetikzlibrary{calc}
\usepackage{xcolor}

% colors
\definecolor{LightBlue}{HTML}{42bbed}
\definecolor{LightGray}{HTML}{616a6b}

% hypertext
\usepackage{hyperref}
\hypersetup{
	colorlinks=true,
	linkcolor=black,
	urlcolor=LightBlue
}

% bibliography
\usepackage{natbib}

% miscellaneous
\usepackage[parfill]{parskip}
\usepackage{nopageno}
\pagestyle{plain}
\usepackage{titling}

% exam class parameters
\bracketedpoints
\pointpoints{b}{b}
\renewcommand{\solutiontitle}{\noindent\textbf{Řešení: }}

% custom macros
\usepackage{../../macros}

% title
\title{\vspace{-3ex}Matematická analýza I (NOFY151) – 1. zápočtový test\vspace{-1ex}}
\author{\vspace{-2ex}}
\date{\vspace{-2ex}22. října 2019}

\begin{document}
\maketitle

\begin{questions}
  \question[2] Ukažte, že pro všechna $n \in \N$ platí rovnost
  \begin{equation*}
    \sum_{k=1}^{n} \frac{1}{\parens{2k - 1} \parens{2k + 1}}
    =
    \frac{n}{2n + 1}.
  \end{equation*}
  Pro důkaz použijte matematickou indukci.
  
  \begin{solution}
    Pro $n = 1$ máme
    \begin{equation*}
      \frac{1}{1 \cdot 3} = \frac{1}{2 + 1}
    \end{equation*}
    a rovnost je tedy splněna. Ukažme, že pokud rovnost platí pro $n \in \N$ platí i pro $n + 1$. 
    \begin{align*}
      \sum_{k=1}^{n + 1} \frac{1}{\parens{2k - 1} \parens{2k + 1}}
      &=
      \frac{1}{\parens{2n + 1} \parens{2n + 3}} 
      + 
      \sum_{k=1}^{n} \frac{1}{\parens{2k - 1} \parens{2k + 1}}
      \\
      &\texteq{IP}
      \frac{1}{\parens{2n + 1} \parens{2n + 3}} 
      +
      \frac{n}{2n + 1}
      =
      \frac{1 + 2n^2 + 3 n}{\parens{2n + 1} \parens{2n + 3}}
      \\
      &=
      \frac{\parens{2n + 1} \parens{n + 1}}{\parens{2n + 1} \parens{2n + 3}}
      =
      \frac{n + 1}{2n + 3}.
    \end{align*}
  \end{solution}
  
  \question[2] Ukažte, že pro všechna $n \in \N$ platí nerovnost
  \begin{equation*}
    \sum_{k=1}^{n} \frac{1}{k}
    \le
    2 \sqrt{n}.
  \end{equation*}
  Pro důkaz použijte matematickou indukci.
  
  \begin{solution}
    Pro $n = 1$ je máme
    \begin{equation*}
      \frac{1}{1} \le 2 \sqrt{1},
    \end{equation*}
    a nerovnost je tedy splněna. Ukažme, že pokud rovnost platí pro $n \in \N$ platí i pro $n + 1$. Podle indukčního předpokladu máme
    \begin{equation*}
      \sum_{k=1}^{n + 1} \frac{1}{k}
      =
      \frac{1}{n + 1} + \sum_{k=1}^{n} \frac{1}{k}
      \textle{IP}
      \frac{1}{n + 1} + 2 \sqrt{n}.
    \end{equation*}
    Stačí tedy dokázat nerovnost
    \begin{equation*}
      \frac{1}{n + 1} + 2 \sqrt{n}
      \le
      2 \sqrt{n + 1}.
    \end{equation*}
    Odečtením $2 \sqrt{n}$ a rozšířením pravé strany o $\sqrt{n + 1} + \sqrt{n}$ dostaneme
    \begin{equation*}
      \frac{1}{n + 1}
      \le
      \frac{2}{\sqrt{n + 1} + \sqrt{n}},
    \end{equation*}
    což po jednoduché úpravě vede na
    \begin{equation*}
      \sqrt{n + 1} + \sqrt{n}
      \le
      2 \parens{n + 1}.
    \end{equation*}
    Poslední nerovnost už však platí, neboť
    \begin{equation*}
      \sqrt{n + 1} + \sqrt{n} 
      \le 
      \sqrt{n + 1} + \sqrt{n + 1}
      =
      2 \sqrt{n + 1}
      \le
      2 \parens{n + 1}.
    \end{equation*}
  \end{solution}
  
  \question[2] Podmnožina reálných čísel $P$ je definována předpisem
  \begin{equation*}
    P \defeq \set{\frac{1 - x}{x}}{x > 0}.
  \end{equation*}
  Nalezněte její minimum, maximum, infimum a supremum (pokud existují). Ověřte z definice příslušných pojmů.
  
  \begin{solution}
    Platí $P = (-1, +\infty)$. (Nakreslete si graf funkce $(1 - x) / x$.)
    
    \begin{itemize}
      \item $\nexists \min P$:
      \begin{equation*}
        \forall m \in P \quad \exists x \in P \quad x < m. \quad \parens{\text{Volme např. $x := \frac{m - 1}{2} \in P$.}}
      \end{equation*}
      \item $\inf P = -1$:
      \begin{equation*}
        \forall x \in P \quad -1 \le x \quad \wedge \quad \forall i > -1 \quad \exists x \in P \quad x < i. \quad \parens{\text{Volme např. $x := \frac{i - 1}{2} \in P$.}}
      \end{equation*}
      \item $\nexists \max P$:
      \begin{equation*}
        \forall M \in P \quad \exists x \in P \quad x > M. \quad \parens{\text{Volme např. $x := 2 M \in P$.}}
      \end{equation*}
      \item $\nexists \sup P$: (stačí ověřit, že neexistuje horní závora)
    \end{itemize}
      \begin{equation*}
        \forall s \in \R \quad \exists x \in P \quad x > s. \quad \parens{\text{Volme např. $x := 2 s \in P$.}}
      \end{equation*}        
  \end{solution}
  
  \question[2] Spočtěte (bez použití l'Hôpitalova pravidla nebo Taylorova rozvoje)
  \begin{equation*}
    \lim_{x \to 0} \frac{x^2}{\sqrt{1 + x \sin x} - \sqrt{\cos x}}.
  \end{equation*}
  
  \begin{solution}
  \begin{align*}
    \lim_{x \to 0} \frac{x^2}{\sqrt{1 + x \sin x} - \sqrt{\cos x}}
    &=
    \lim_{x \to 0} \frac{x^2}{\sqrt{1 + x \sin x} - \sqrt{\cos x}} \cdot \frac{\sqrt{1 + x \sin x} + \sqrt{\cos x}}{\sqrt{1 + x \sin x} + \sqrt{\cos x}}
    \\
    &=
    \lim_{x \to 0} \frac{x^2}{1 + x \sin x - \cos x} \cdot \lim_{x \to 0} \parens{\sqrt{1 + x \sin x} + \sqrt{\cos x}}
    \\
    &=
    \lim_{x \to 0} \frac{1}{\frac{\sin x}{x} + \frac{1 - \cos x}{x^2}} \cdot 2
    =
    \frac{1}{1+\frac{1}{2}} \cdot 2 = \frac{4}{3},
  \end{align*}
  kde jsme využili znalost limit
  \begin{equation*}
    \lim_{x \to 0} \frac{\sin x}{x} = 1, \qquad \lim_{x \to 0} \frac{1 - \cos x}{x^2} = \frac{1}{2}.
  \end{equation*}
  \end{solution}
  
  \question[2] Pro $n \in \N$ spočtěte (bez použití l'Hôpitalova pravidla nebo Taylorova rozvoje)
  \begin{equation*}
    \lim_{x \to 1} 
    \frac{
      \parens{1 - \sqrt{x}} \parens{1 - \sqrt[3]{x}} \cdots \parens{1 - \sqrt[n]{x}}
    }{
      \parens{1 - x}^{n - 1}
    }.
  \end{equation*}
  
  \begin{solution}
    \begin{align*}
      \lim_{x \to 1} 
      \frac{
        \parens{1 - \sqrt{x}} \parens{1 - \sqrt[3]{x}} \cdot \ldots \cdot \parens{1 - \sqrt[n]{x}}
      }{
        \parens{1 - x}^{n - 1}
      }
      &=
      \lim_{x \to 1} \frac{1 - \sqrt{x}}{1 - x} \cdots \lim_{x \to 1} \frac{1 - \sqrt[n]{x}}{1 - x}
      \\
      &=
      \lim_{x \to 1} \frac{\sqrt{x} - 1}{x - 1} \cdots \lim_{x \to 1} \frac{\sqrt[n]{x} - 1}{x - 1}
      \\
      &=
      \frac{1}{2} \cdots \frac{1}{n} = \frac{1}{n!},
    \end{align*}
    kde jsme v poslední rovnosti využili znalost limity
    \begin{equation*}
      \lim_{y \to 0} \frac{\sqrt[n]{1 + y} - 1}{y} = \frac{1}{n},
    \end{equation*}
    a substituci $y = x - 1$.
    
  \end{solution}
   
\end{questions}

\end{document}
