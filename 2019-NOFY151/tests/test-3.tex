\documentclass[answers]{exam}

% encoding, language
\usepackage[T1]{fontenc}
\usepackage[czech]{babel}
\usepackage[utf8]{inputenc}

% geometry
\usepackage[a4paper]{geometry}
\geometry{
  a4paper,
  total={170mm,257mm},
  left=20mm,
  top=20mm}

% mathematics
\usepackage{amsmath}
\usepackage{amssymb}
\usepackage{amsthm}

% text appearence
\usepackage{libertine}
\usepackage{microtype}  % better general appearence of text

% graphics
\usepackage{graphicx}
\graphicspath{{figures/}}
\usepackage{tikz}
\usepackage{pgfplots}
\usetikzlibrary{calc}
\usepackage{xcolor}

% colors
\definecolor{LightBlue}{HTML}{42bbed}
\definecolor{LightGray}{HTML}{616a6b}
\definecolor{GraphColor}{HTML}{1d7ad1}

% hypertext
\usepackage{hyperref}
\hypersetup{
	colorlinks=true,
	linkcolor=black,
	urlcolor=LightBlue
}

% bibliography
\usepackage{natbib}

% miscellaneous
\usepackage[parfill]{parskip}
\usepackage{nopageno}
\pagestyle{plain}
\usepackage{titling}

% exam class parameters
\bracketedpoints
\pointpoints{b}{b}
\renewcommand{\solutiontitle}{\noindent\textbf{Řešení: }}

% custom macros
\usepackage{../../macros}

% title
\title{\vspace{-3ex}Matematická analýza I (NOFY151) – 3. zápočtový test\vspace{-1ex}}
\author{\vspace{-2ex}}
\date{\vspace{-2ex}7. ledna 2020}

\begin{document}
\maketitle

Při vyšetřování průběhu funkce nás zajímá:
\begin{enumerate}
	\item definiční obor
	\item obor spojitosti 
	\item sudost/lichost, periodicita, jiné symetrie
	\item limity v krajních bodech definičního oboru (či jeho podintervalů) a v bodech nespojitosti
	\item průsečíky s osami
	\item první derivace (i jednostranné derivace v bodech, které derivaci nemají)
	\item druhá derivace
	\item vyhodnocení první a druhé derivace: 
		\begin{enumerate}
			\item množiny monotonie
			\item lokální a globální extrémy
			\item obor hodnot
			\item konvexita, konkávnost, inflexní body
		\end{enumerate} 
	\item asymptoty
	\item náčrt grafu
\end{enumerate}

\vspace{6pt}
\hrule
\vspace{6pt}

\begin{questions}
  \question[4] Vyšetřete průběh funkce
  \begin{equation*}
    f(x) \defeq \frac{x}{\ln x}.
  \end{equation*}
  
  \begin{solution}
    \begin{enumerate}
    	\item Jmenovatel zadané funkce je definovaný a nenulový pro všechna $x \in \parens{0, 1} \cup \parens{1, +\infty}$, tedy $D_f = \parens{0, 1} \cup \parens{1, +\infty}$.
    	
    	\item Protože čitatel i jmenovatel zadané funkce jsou spojité funkce na $D_f$, na stejné množině je tedy spojitá i celá funkce $f$ (podíl spojitých funkcí je spojitá funkce).
    	
    	\item Funkce není definovaná pro záporné hodnoty nemůže být tedy ani sudá ani lichá. Taktéž není periodická a nevykazuje ani jiné symetrie.
    	
    	\item \label{item:4}
    		Limity v krajních bodech definičního oboru vycházejí následovně
    		\begin{align*}
    			\lim_{x \to 0+} \frac{x}{\ln x}
    			&\texteq{l'H}
    			\lim_{x \to 0+} \frac{1}{\frac{1}{x}}
    			=
    			0,
    			\\
    			\lim_{x \to +\infty} \frac{x}{\ln x}
    			&\texteq{l'H}
    			\lim_{x \to +\infty} \frac{1}{\frac{1}{x}}
    			=
    			+\infty,
    			\\
    			\lim_{x \to 1-} \frac{x}{\ln x}
    			&= - \infty,
    			\\
    			\lim_{x \to 1+} \frac{x}{\ln x}
    			&= + \infty,
    		\end{align*}
    		kde jsme v prvních dvou případech využili l'Hôpitalovo pravidlo pro výpočet limity typu ``$\frac{\textrm{něco}}{\infty}$''.
    	\item 
    		Protože funkce není definovaná v nule, nemá průsečík s osou $y$. S osou $x$ taktéž neexistuje žádný průsečík, protože čitatel zadané funkce by v takovém případě musel být nulový, což opět vylučuje definiční obor funkce.
    		
    		\item První derivace je rovna
    			\begin{equation*}
    				\dd{f}{x}(x)
    				=
    				\frac{\ln x - x \cdot \frac{1}{x}}{\ln^2 x}
    				=
    				\frac{\ln x - 1}{\ln^2 x}.
    			\end{equation*}
    			Všimněme si, že $D_{\dd{f}{x}} = D_f$.
					
					V bodě $0$ není funkce $f$ definovaná, nemůže tam mít tedy ani jednostrannou derivaci. Můžeme ale jako dodatečnou informaci spočítat limitu první derivace pro $x$ jdoucí k nule zprava
    			\begin{equation*}
    				\lim_{x \to 0+} \dd{f}{x}(x)
    				=
    				\lim_{x \to 0+} \frac{\ln x - 1}{\ln^2 x}
    				\texteq{l'H}
    				\lim_{x \to 0+} \frac{\frac{1}{x}}{2 \ln x \cdot \frac{1}{x}}
    				=
    				\lim_{x \to 0+} \frac{1}{2 \ln x}
    				=
    				0.
    			\end{equation*}
    		
    		\item Druhá derivace je rovna
    			\begin{equation*}
    				\dd[2]{f}{x}(x)
    				=
    				\frac{\frac{1}{x}\ln^2 x - \parens{\ln x - 1} \cdot 2 \ln x \cdot \frac{1}{x}}{\ln^4 x}
    				=
    				\frac{2 - \ln x}{x \ln^3 x}.
    			\end{equation*}
    			Všimněme si, že $D_{\dd[2]{f}{x}} = D_f$.
    			
    		\item Z podmínky nulové první derivace dostaneme kandidáty na lokální extrémy
    		\begin{equation*}
    			\dd{f}{x}(x) = 0 \iff x = \exponential{}.
    		\end{equation*}
    		Protože $D_{\dd{f}{x}} = D_f$, první derivace existuje všude a nedostáváme tak žádné další kandidáty na extrém.
    		
    		Z podmínky nulové druhé derivace dostaneme kandidáty na inflexní body
    		\begin{equation*}
    			\dd[2]{f}{x}(x) = 0 \iff x = \exponential{2}.
    		\end{equation*}
    		Protože $D_{\dd[2]{f}{x}} = D_f$, první derivace existuje všude a nedostáváme tak žádné další kandidáty na inflexní body.
    		    		
    		Definiční obor funkce tak můžeme podle významných bodů rozdělit na $4$ intervaly a určit funkční hodnoty v těchto bodech a znaménka derivací na příslušných intervalech
    		\begin{center}
	    		\begin{tabular}{c|ccccccccc}
	    			& $0$ & $\parens{0, 1}$ & $1$ & $\parens{1, \exponential{}}$ & $\exponential{}$ & $\parens{\exponential{}, \exponential{2}}$ & $\exponential{2}$ & $\parens{\exponential{2}, +\infty}$ & $+\infty$ \\[2pt]
	    			\hline \\[-8pt]
	    			$f$ & $0$ & & $\mp \infty$ & & $\exponential{}$ & & $\frac{\exponential{2}}{2}$ & & $+\infty$ \\[3pt]
	    			$\dd{f}{x}$ & $0$ & $-$ & & $-$ & $0$ & $+$ & & $+$ & \\[3pt]
	    			$\dd[2]{f}{x}$ & & $-$ & & $+$ & & $+$ & $0$ & $-$ &
	    		\end{tabular}
    		\end{center}
    		(Zde rozumíme \uv{funkční hodnotou} v bodech $0$, $1$ a $+\infty$ (jednostrannou) limitu příslušné funkce v těchto bodech.)
    		
    		Z tabulky výše pak dostáváme
    		\begin{enumerate}
    			\item funkce je klesající na intervalech $\parens{0, 1}$ a $\parens{1, \exponential{}}$ a rostoucí na intervalu $\parens{\exponential{}, +\infty}$
    			\item v bodě $\exponential{}$ má funkce lokální minimum, globální extrémy funkce nemá
    			\item $R_f = \parens{-\infty, 0} \cup [\exponential{}, +\infty)$
    			\item funkce je konvexní na intervalu $\parens{1,\exponential{2}}$, konkávní na intervalech $\parens{0, 1}$ a $\parens{\exponential{2}, +\infty}$ a v bodě $\exponential{2}$ má inflexní bod
    		\end{enumerate}
    		
    		\item Podle bodu \ref{item:4} má funkce vertikální asymptotu v bodě $1$. Pro ověření zda má funkce asymptotu v nekonečnu vypočítejme, jakou by musela mít směrnici $k$
    		\begin{equation*}
    			k 
    			= 
    			\lim_{x \to +\infty} \frac{f(x)}{x}
    			=
    			\lim_{x \to +\infty} \frac{1}{\ln x}
    			=
    			0. 
    		\end{equation*}
    		Asymptota v nekonečnu by pak musela být horizontální přímka a muselo by tedy platit
    		\begin{equation*}
    		 \lim_{x \to +\infty} f(x) \in \R.
    		\end{equation*}
    		To je ale ve sporu s bodem \ref{item:4}. Asymptotu v nekonečnu tedy funkce $f$ nemá.
    		
    		\item Graf funkce $f$ vypadá následovně 
		    \begin{center}
		      \begin{tikzpicture}
		        \begin{axis}[
	    	      width=0.85\textwidth,
	    	      height=0.57\textwidth,
		          xlabel=$x$, 
		          xmin=0,
		          xmax=15,
		          xtick pos=bottom,
		          xtick={0,1,2.7182818,7.3890559},
		          xticklabels={$0$,$1$,$\exponential{\phantom{1}}$,$\exponential{2}$},
		          ylabel=$f(x)$,
		          ylabel style={rotate=-90, xshift=7pt},
		          ymin=-6,
		          ymax=10,
		          ytick pos=left,
		          ytick={0,2.7182818,3.69452795},
		          yticklabels={$0$,$\exponential{\phantom{1}}$,$\frac{\exponential{2}}{2}$},]
		          \addplot[
		            blue, samples=1000, smooth, color=GraphColor, line width=0.8pt, domain=0.001:0.999
		          ]
		            {x / ln(x)};
		          \addplot[
		            blue, samples=1000, smooth, color=GraphColor, line width=0.8pt, domain=1.01:15
		          ]{x / ln(x)};		
              \addplot[
                only marks, mark size=1.5pt, color=GraphColor, samples at = {2.7182818,7.3890559}
              ]{x / ln(x)};
              \addplot[
                only marks, mark size=1.5pt, mark options={fill=white}, color=GraphColor, samples at = {0}
              ]{0};
		          \addplot[dashed] coordinates {(1,-6) (1,10)};
		          \addplot[dotted] coordinates {(2.7182818,-6) (2.7182818,2.7182818)};
		          \addplot[dotted] coordinates {(0,2.7182818) (2.7182818,2.7182818)};
		          \addplot[dotted] coordinates {(7.3890559,-6) (7.3890559,3.69452795)};
		          \addplot[dotted] coordinates {(0,3.69452795) (7.3890559,3.69452795)};
		        \end{axis}
		      \end{tikzpicture}
		    \end{center}
    \end{enumerate}
  \end{solution}
  
  \question[6] Vyšetřete průběh funkce
	\begin{equation*}
	f(x) \defeq 2 \abs{\sin x} + \abs{\cos 2x}.
	\end{equation*}
	
  \begin{solution}
    \begin{enumerate}
    	\item Funkce je definovaná pro všechna $x \in \R$, tedy $D_f = \R$.
    	
    	\item Goniometrické funkce $\sin$ a $\cos$ jsou spojité na $\R$, absolutní hodnota též a jejich složení a součet taktéž. Celkově je tedy obor spojitosti funkce $f$ celá reálná osa.
    	
    	\item Funkce $\abs{\sin x}$ je $\pi$-periodická a funkce $\abs{\cos 2x}$ je $\frac{\pi}{2}$-periodická (nakreslete si obrázky). Celkově je tedy zadaná funkce $\pi$-periodická. Navíc platí
    	\begin{equation*}
    		f(-x) = 2 \abs{\sin(-x)} + \abs{\cos(-x)} = 2 \abs{- \sin x} + \abs{\cos x} = 2 \abs{\sin x} + \abs{\cos 2x} = f(x).
    	\end{equation*}
    	Zadaná funkce je tedy sudá. Z těchto pozorování plyne, že se můžeme při vyšetřování průběhu funkce omezit na interval $[0, \frac{\pi}{2}]$. Na tomto intervalu pak můžeme funkci $f$ psát jako
    	\begin{equation}
    		\label{eq:1}
    		f(x)
    		=
    		\begin{cases}
    			2 \sin x + \cos 2x, &\quad \text{na $\brackets{0, \frac{\pi}{4}}$,}
    			\\
    			2 \sin x - \cos 2x, &\quad \text{na $\left( \frac{\pi}{4}, \frac{\pi}{2} \right]$.}
    		\end{cases}
    	\end{equation}
    	
    	\item \label{item:4b}
    		Zadaná funkce je spojitá na celém $\R$, tedy i v krajních bodech intervalu $[0, \frac{\pi}{2}]$. Stačí tedy spočítat funkční hodnoty v těchto bodech
    		\begin{align*}
    			f(0)
    			&=
    			1,
    			\\
    			f\parens{\frac{\pi}{2}}
    			&=
    			3.
    		\end{align*}
    	\item 
    		Průsečík s osou $y$ už máme z předchozího bodu: $f(0) = 1$. Abychom dostali průsečík s osou $x$, je potřeba vyřešit rovnici
    		\begin{equation*}
    			2 \abs{\sin x} + \abs{\cos 2x} = 0.
    		\end{equation*}
    		Protože absolutní hodnota je vždy nezáporná, vidíme, že rovnost může nastat pouze, pokud
    		\begin{equation*}
	    		\sin x = 0, \quad \land \quad \cos 2x = 0,
    		\end{equation*}
    		což je na intervalu $[0, \frac{\pi}{2}]$ ekvivalentní
    		\begin{equation*}
    			x = 0, \quad \land \quad x = \frac{\pi}{4}.
    		\end{equation*}
    		To ale nemůže nastat a funkce tak žádný průsečík s osou $x$ nemá.
    		
    		\item \label{item:6} První derivace je rovna (využíváme zápisu funkce $f$ ve tvaru \eqref{eq:1})
		    	\begin{equation*}
		    		\dd{f}{x}(x)
		    		=
		    		\begin{cases}
		    			2 \cos x - 2 \sin 2x, &\quad \text{na $\parens{0, \frac{\pi}{4}}$,}
		    			\\
		    			2 \cos x + 2 \sin 2x, &\quad \text{na $\parens{\frac{\pi}{4}, \frac{\pi}{2}}$.}
		    		\end{cases}
		    	\end{equation*}
					Jednostranné derivace v krajních bodech intervalů $(0, \frac{\pi}{4})$ a $(\frac{\pi}{4}, \frac{\pi}{2})$ vychází následovně
					\begin{align*}
						\dd{f_+}{x}(0) 
						&=
						\lim_{x \to 0+} \dd{f}{x}(x) 
						= 
						\lim_{x \to 0+} \parens{2 \cos x - 2 \sin 2x}
						=
						2,
						\\
						\dd{f_-}{x}\parens{\frac{\pi}{4}} 
						&=
						\lim_{x \to \frac{\pi}{4}-} \dd{f}{x}(x) 
						= 
						\lim_{x \to \frac{\pi}{4}-} \parens{2 \cos x - 2 \sin 2x}
						=
						\sqrt{2} - 2,			
						\\
						\dd{f_+}{x}\parens{\frac{\pi}{4}} 
						&=
						\lim_{x \to \frac{\pi}{4}+} \dd{f}{x}(x) 
						= 
						\lim_{x \to \frac{\pi}{4}+} \parens{2 \cos x + 2 \sin 2x}
						=
						\sqrt{2} + 2,
						\\
						\dd{f_-}{x}\parens{\frac{\pi}{2}} 
						&=
						\lim_{x \to \frac{\pi}{2}-} \dd{f}{x}(x) 
						= 
						\lim_{x \to \frac{\pi}{2}-} \parens{2 \cos x + 2 \sin 2x}
						=
						0.								
					\end{align*}
					   		
    		\item Druhá derivace je rovna (opět využíváme zápisu funkce $f$ ve tvaru \eqref{eq:1})
		    	\begin{equation*}
		    		\dd[2]{f}{x}(x)
		    		=
		    		\begin{cases}
		    			- 2 \sin x - 4 \cos 2x, &\quad \text{na $\parens{0, \frac{\pi}{4}}$,}
		    			\\
		    			- 2 \sin x + 4 \cos 2x, &\quad \text{na $\parens{\frac{\pi}{4}, \frac{\pi}{2}}$.}
		    		\end{cases}
		    	\end{equation*}
    			
    		\item Z podmínky nulové první derivace ve vnitřních bodech intervalů, kde derivace existuje, dostaneme kandidáty na lokální extrémy. Na intervalu $(0, \frac{\pi}{4})$ by mělo platit
    		\begin{equation*}
    			2 \cos x - 2 \sin 2x = 0,
    		\end{equation*}
    		což se s využitím vzorce $\sin 2x = 2 \sin x \cos x$ a nenulovosti $\cos x$ na intervalu $(0, \frac{\pi}{4})$ redukuje na rovnici
    		\begin{equation*}
    			\sin x = \frac{1}{2}.
    		\end{equation*}
    		Dostáváme tak kandidáta na extrém v bodě $x = \frac{\pi}{6}$.
    		
    		Na intervalu $(\frac{\pi}{4}, \frac{\pi}{2})$ by mělo platit
    		\begin{equation*}
    			2 \cos x + 2 \sin 2x = 0,
    		\end{equation*}
    		což se opět s využitím vzorce $\sin 2x = 2 \sin x \cos x$ a nenulovosti $\cos x$ na intervalu $(\frac{\pi}{4}, \frac{\pi}{2})$ redukuje na rovnici
    		\begin{equation*}
    			\sin x = -\frac{1}{2}.
    		\end{equation*}    		
    		To ale není splněno pro žádné $x$ z intervalu $(\frac{\pi}{4}, \frac{\pi}{2})$.
    		
    		Podle bodu \ref{item:6} dostáváme ještě kandidáta na extrém v bodě $\frac{\pi}{4}$, neboť zde neexistuje první derivace (jednostranné derivace v tomto bodě se různí).
    		    		
    		Z podmínky nulové druhé derivace ve vnitřních bodech intervalů, kde derivace existuje, dostaneme kandidáty na inflexní body. Na intervalu $(0, \frac{\pi}{4})$ by mělo platit
    		\begin{equation*}
    			- 2 \sin x - 4 \cos 2x = 0.
    		\end{equation*}
    		Na tomto intervalu jsou ovšem $\sin x$ i $\cos 2x$ kladné, takže rovnost nemůže nastat.
    		
    		Na intervalu $(\frac{\pi}{4}, \frac{\pi}{2})$ by mělo platit
    		\begin{equation*}
    			- 2 \sin x + 4 \cos 2x = 0.
    		\end{equation*}
    		Na tomto intervalu je ovšem $\sin x$ kladný a $\cos 2x$ záporný, takže rovnost opět nemůže nastat.
    		
    		Jak už víme, v bodě $\frac{\pi}{4}$ neexistuje první derivace a v takovém bodě tedy nemůže být ani inflexní bod (funkce musí být v inflexním bodě diferencovatelná). Žádní kandidáti na inflexní bod tedy neexistují.
    		
    		Interval $[0, \frac{\pi}{2}]$ tak můžeme podle významných bodů rozdělit na $3$ intervaly a určit funkční hodnoty v těchto bodech a znaménka derivací na příslušných intervalech
    		\begin{center}
	    		\begin{tabular}{c|ccccccc}
	    			& $0$ & $\parens{0, \frac{\pi}{6}}$ & $\frac{\pi}{6}$ & $\parens{\frac{\pi}{6}, \frac{\pi}{4}}$ & $\frac{\pi}{4}$ & $\parens{\frac{\pi}{4}, \frac{\pi}{2}}$ & $\frac{\pi}{2}$ \\[2pt]
	    			\hline \\[-8pt]
	    			$f$ & $1$ & & $\frac{3}{2}$ & & $\sqrt{2}$ & & $3$  \\[3pt]
	    			$\dd{f}{x}$ & $2$ & $+$ & $0$ & $-$ & $\sqrt{2} \mp 2$ & $+$ & $0$ \\[3pt]
	    			$\dd[2]{f}{x}$ & & $-$ & & $-$ & & $-$
	    		\end{tabular}
    		\end{center}
    		(Zde rozumíme \uv{funkční hodnotou} první derivace v bodech $0$, $\frac{\pi}{4}$ a $\frac{\pi}{2}$ jednostrannou derivaci v příslušných bodech.)
    		
    		Z tabulky výše pak pro funkci $f$ na intervalu $[0, \frac{\pi}{2}]$ dostáváme
    		\begin{enumerate}
    			\item \label{item:a} funkce je klesající na intervalu $\brackets{\frac{\pi}{6}, \frac{\pi}{4}}$ a rostoucí na intervalech $\brackets{0, \frac{\pi}{6}}$ a $\brackets{\frac{\pi}{4}, \frac{\pi}{2}}$
    			\item \label{item:b} v bodě $\frac{\pi}{4}$ má funkce lokální minimum, v bodě $\frac{\pi}{6}$  lokální maximum, v bodě $0$ globální minimum a v bodě $\frac{\pi}{2}$ globální maximum
    			\item $R_f = \brackets{1, 3}$
    			\item \label{item:c} funkce je konkávní na intervalech $\brackets{0, \frac{\pi}{4}}$ a $\brackets{\frac{\pi}{4}, \frac{\pi}{2}}$, inflexní body nemá
    		\end{enumerate}
    		
    		Vlastnosti z bodů \ref{item:a}, \ref{item:b} a \ref{item:c} by se s využitím sudosti a periodicity funkce $f$ snadno rozšířily na celý definiční obor zadané funkce.
    		
    		\item Zadaná funkce je spojitá na celém $\R$, takže nemá žádné vertikální asymptoty. Pro ověření zda má funkce asymptotu v nekonečnu vypočítejme, jakou by musela mít směrnici $k$
    		\begin{equation*}
    			k 
    			= 
    			\lim_{x \to +\infty} \frac{f(x)}{x}
    			=
    			0,
    		\end{equation*}
    		neboť $f$ je omezená a $\frac{1}{x}$ jde k nule pro $x \to +\infty$.
    		Asymptota v nekonečnu by pak musela být horizontální přímka a muselo by tedy platit
    		\begin{equation*}
					\lim_{x \to +\infty} f(x) \in \R.
    		\end{equation*}
    		Protože je ale $f$ periodická a nekonstantní, z Heineho věty plyne, že její limita v nekonečnu neexistuje. Asymptotu v nekonečnu tedy funkce $f$ nemá.
    		
    		\item Graf funkce $f$ vypadá následovně 
		    \begin{center}
		      \begin{tikzpicture}
		        \begin{axis}[
	    	      width=0.92\textwidth,
	    	      height=0.72\textwidth,
		          xlabel=$x$, 
		          xmin=-3.14159,
		          xmax=3.14159,
		          xtick pos=bottom,
		          xtick={-3.14159,0,0.5235987,0.785398,1.57079632679,2.35619449019,2.617993877,3.14159},
		          xticklabels={$-\pi$,$0$,$\frac{\pi}{6}$,$\frac{\pi}{4}$,$\frac{\pi}{2}$,$\frac{3\pi}{4}$,$\frac{5\pi}{6}$,$\pi$},
		          ylabel=$f(x)$,
		          ylabel style={rotate=-90, xshift=7pt},
		          ymin=0.95,
		          ymax=3.05,
		          ytick pos=left,
		          ytick={1,1.41421,1.5, 3},
		          yticklabels={$1$,$\sqrt{2}$,$\frac{3}{2}$,$3$},]
		          \addplot[
		            blue, samples=1000, smooth, color=GraphColor, line width=0.8pt, domain=0:1.57079632679
		          ]
		            {abs(2*sin(deg(x))) + abs(cos(2*deg(x)))};
		          \addplot[
		            blue, samples=1000, smooth, color=GraphColor, line width=0.4pt, domain=1.57079632679:3.14159
		          ]
		            {abs(2*sin(deg(x))) + abs(cos(2*deg(x)))};
		          \addplot[
		            blue, samples=1000, smooth, color=GraphColor, line width=0.4pt, domain=-3.14159:0
		          ]
		            {abs(2*sin(deg(x))) + abs(cos(2*deg(x)))};	
              \addplot[
                only marks, mark size=1.5pt, color=GraphColor, samples at = {0,0.5235987,0.785398,1.57079632679}
              ]
              	{abs(2*sin(deg(x))) + abs(cos(2*deg(x)))};
		          \addplot[dotted] coordinates {(0,0.8) (0,1)};
		          \addplot[dotted] coordinates {(-3.14159,1) (0,1)};
		          \addplot[dotted] coordinates {(0.5235987,0.8) (0.5235987,1.5)};
		          \addplot[dotted] coordinates {(-3.14159,1.5) (0.5235987,1.5)};
		          \addplot[dotted] coordinates {(0.785398,0.8) (0.785398,1.41421)};
		          \addplot[dotted] coordinates {(-3.14159,1.41421) (0.785398,1.41421)};
		          \addplot[dotted] coordinates {(1.57079632679,0.8) (1.57079632679,3)};
		          \addplot[dotted] coordinates {(-3.14159,3) (1.57079632679,3)};		          
		        \end{axis}
		      \end{tikzpicture}
		    \end{center}
		    Zvýrazněná část grafu odpovídá zkoumané funkci na intervalu $[0, \frac{\pi}{2}]$, kde jsme vyšetřovali její průběh.
    \end{enumerate}
  \end{solution}
	   
\end{questions}

\end{document}
