\documentclass[answers]{exam}

% encoding, language
\usepackage[T1]{fontenc}
\usepackage[czech]{babel}
\usepackage[utf8]{inputenc}

% geometry
\usepackage[a4paper]{geometry}
\geometry{
  a4paper,
  total={170mm,257mm},
  left=20mm,
  top=20mm}

% mathematics
\usepackage{amsmath}
\usepackage{amssymb}
\usepackage{amsthm}

% text appearence
\usepackage{libertine}
\usepackage{microtype}  % better general appearence of text

% graphics
\usepackage{graphicx}
\graphicspath{{figures/}}
\usepackage{tikz}
\usepackage{pgfplots}
\usetikzlibrary{calc}
\usepackage{xcolor}

% colors
\definecolor{LightBlue}{HTML}{42bbed}
\definecolor{LightGray}{HTML}{616a6b}
\definecolor{MarkColor}{HTML}{1d7ad1}

% hypertext
\usepackage{hyperref}
\hypersetup{
	colorlinks=true,
	linkcolor=black,
	urlcolor=LightBlue
}

% bibliography
\usepackage{natbib}

% miscellaneous
\usepackage[parfill]{parskip}
\usepackage{nopageno}
\pagestyle{plain}
\usepackage{titling}

% exam class parameters
\bracketedpoints
\pointpoints{b}{b}
\renewcommand{\solutiontitle}{\noindent\textbf{Řešení: }}

% custom macros
\usepackage{../../macros}

% title
\title{\vspace{-3ex}Matematická analýza I (NOFY151) – 2. zápočtový test (opravný)\vspace{-1ex}}
\author{\vspace{-2ex}}
\date{\vspace{-2ex}10. prosince 2019}

\begin{document}
\maketitle

\begin{questions}
  \question[2] Určete definiční obor funkce
  \begin{equation*}
    f(x) \defeq \sqrt{x - \sqrt{x}},
  \end{equation*}
  a v jeho vnitřních bodech spočtěte derivaci $f$.
  
  \begin{solution}
    Kvůli \uv{vnitřní} odmocnině musí být nutně $x \ge 0$. Dále argument \uv{vnější} odmocniny musí být nezáporný, tj.
    \begin{equation*}
      x - \sqrt{x} \ge 0.
    \end{equation*}
    Po přehození $\sqrt{x}$ na pravou stranu a umocnění nerovnosti dostáváme
    \begin{equation*}
      x^2 \ge x,
    \end{equation*}
    což je ekvivaletní s
    \begin{equation*}
      x \parens{x - 1} \ge 0.
    \end{equation*}
    Z poslední nerovnosti pak plyne, že $D_f = \left[ 1, +\infty \right)$.
    
    Ve vnitřních bodech definičního oboru, tj. na intervalu $(1, +\infty)$, najdeme derivaci funkce $f$ pomocí věty o derivaci složené funkce
    \begin{equation*}
      \dd{f}{x}(x) 
      = 
      \dd{}{x} \parens{\parens{x - x^{\frac{1}{2}}}^{\frac{1}{2}}}
      =
      \frac{1}{2} \parens{x - x^{\frac{1}{2}}}^{-\frac{1}{2}} \cdot \parens{1 - \frac{1}{2} x^{-\frac{1}{2}}}
      =
      \frac{1}{2\sqrt{x - \sqrt{x}}} \cdot \parens{1 - \frac{1}{2 \sqrt{x}}}
      =
      \frac{2 \sqrt{x} - 1}{4 \sqrt{x\parens{x - \sqrt{x}}}}.
    \end{equation*}
  \end{solution}
  
  \question[4] Spočtěte
  \begin{equation*}
    \int \ln \parens{\sqrt{x + 1} - \sqrt{x - 1}} \diff x
  \end{equation*}
  na maximálním možném intervalu (a ten určete).
  
  \begin{solution}
    Předně argument logaritmu musí být kladný, tj.    
    \begin{equation*}
      \sqrt{x + 1} - \sqrt{x - 1} > 0.
    \end{equation*}
    Po přehození členu $\sqrt{x - 1}$ na pravou stranu a umocnění nerovnosti dostáváme
    \begin{equation*}
      x + 1 > x - 1.
    \end{equation*}
    To je ovšem splněno pro všechna $x \in \R$. Další omezení plyne z požadavku nezáporného argumentu odmocnin, které dává $x \ge -1$ a $x \ge 1$. Definiční obor integrandu je tedy $[1, +\infty)$, což je interval, na kterém je integrand spojitý a bude na něm tedy mít primitivní funkci.
    
    Protože budeme chtít pro výpočet integrálu použít metodu per partes, spočtěme si nejdřív derivaci integrandu
    \begin{align*}
      \dd{}{x} \ln \parens{\sqrt{x + 1} - \sqrt{x - 1}}
      &=
      \frac{1}{\sqrt{x + 1} - \sqrt{x - 1}}
      \cdot
      \left( \frac{1}{2 \sqrt{x + 1}} - \frac{1}{2 \sqrt{x - 1}} \right)
      \\
      &=
      \frac{1}{\sqrt{x + 1} - \sqrt{x - 1}}
      \cdot
      \frac{\sqrt{x - 1} - \sqrt{x + 1}}{2 \sqrt{x + 1} \sqrt{x - 1}}
      \\
      &=
      -\frac{1}{2\sqrt{x^2 - 1}}.
    \end{align*}
    Nyní již můžeme počítat
    \begin{align*}
      \int \ln \parens{\sqrt{x + 1} - \sqrt{x - 1}} \diff x
      &=
      \left| 
        \begin{aligned}
          u &= \ln \parens{\sqrt{x + 1} - \sqrt{x - 1}} & v &= x
          \\
          u' &= -\frac{1}{2\sqrt{x^2 - 1}} & v' &= 1
        \end{aligned}
      \right|
      \\
      &=
      x \ln \parens{\sqrt{x + 1} - \sqrt{x - 1}}
      +
      \int \frac{x}{2\sqrt{x^2 - 1}} \diff x.
    \end{align*}
    Poslední integrál vyřešíme první substituční metodou
    \begin{equation*}
      \int \frac{x}{2\sqrt{x^2 - 1}} \diff x
      =
      \left| 
        \begin{aligned}
          t &= x^2 - 1
          \\
          \diff t & = 2x \diff x
        \end{aligned}
      \right|
      =
      \frac{1}{4} \int t^{-\frac{1}{2}} \diff t
      =
      \frac{1}{2} \sqrt{t} + c
      =
      \frac{1}{2} \sqrt{x^2 - 1} + c.
    \end{equation*}
    Celkově tedy dostáváme
    \begin{equation*}
      \int \ln \parens{\sqrt{x + 1} - \sqrt{x - 1}} \diff x
      =
      x \ln \parens{\sqrt{x + 1} - \sqrt{x - 1}}
      +
      \frac{1}{2} \sqrt{x^2 - 1} 
      + c.
    \end{equation*}
  \end{solution}
  
%  \question[3] Spočtěte
%  \begin{equation*}
%    \int \frac{\tan \frac{1}{x}}{x^2} \diff x,
%  \end{equation*}
%  na maximálních možných intervalech (a ty určete).
%  
%  \begin{solution}
%    Integrand je definovaný na intervalech
%    \begin{equation*}
%      \int \frac{\tan \frac{1}{x}}{x^2} \diff x
%      =
%      \left| 
%        \begin{aligned}
%          u &= \frac{1}{x}
%          \\
%          \diff u &= - \frac{1}{x^2} \diff x
%        \end{aligned}
%      \right|
%      =
%      - \int \tan u \diff u
%      =
%      - \int \frac{\sin u}{\cos u} \diff u
%      =
%      \left|
%        \begin{aligned}
%          v &= \cos u
%          \\
%          \diff v &= - \sin u \diff u
%        \end{aligned}
%      \right|
%      =
%      \int \frac{1}{v} \diff v
%      =
%      \ln \abs{v} + c      
%    \end{equation*}
%  \end{solution}
  
%  \question[2] Spočtěte limitu funkce
%  \begin{equation*}
%    \lim_{x \to 1} \frac{x^x - x}{\ln x - x + 1}.
%  \end{equation*}
%  
%  \begin{solution}
%  Člen $x^x$ si přepíšeme jako $\exponential{x \ln x}$ a počítáme
%  \begin{align*}
%    \lim_{x \to 1} \frac{\exponential{x \ln x} - x}{\ln x - x + 1}
%    \texteq{l'H}
%    \lim_{x \to 1} \frac{\exponential{x \ln x} \parens{\ln x + 1} - 1}{\frac{1}{x} - 1}
%    \texteq{l'H}
%    \lim_{x \to 1} \frac{\exponential{x \ln x} \parens{\ln x + 1}^2 + \exponential{x \ln x} \frac{1}{x}}{-\frac{1}{x^2}}
%    =
%    -2,
%  \end{align*}
%  kde jsme v prvních dvou rovnostech využili l'Hôpitalova pravidla pro výpočet limity typu ``$\frac{0}{0}$''.
%  \end{solution}

  \question[2] Spočtěte
  \begin{equation*}
    \int \frac{x^3}{\parens{x - 1}^{100}} \diff x
  \end{equation*}
  na maximálních možných intervalech (a ty určete).
  
  \begin{solution}
  Integrand je zjevně definovaný a spojitý na intervalech $(-\infty, 1)$ a $(1, +\infty)$. Na těchto intervalech tedy budeme hledat primitivní funkci. 
  
  Byť je integrand ve tvaru vhodném k rozkladu na parciální zlomky, byla by chyba tento postup použít, protože by to vedlo k hledání $100$ konstant. Místo toho použijeme jednoduchou substituci
  \begin{align*}
    \int \frac{x^3}{\parens{x - 1}^{100}} \diff x
    &=
    \left| 
      \begin{aligned}
        t &= x - 1
        \\
        \diff t & = \diff x
      \end{aligned}
    \right|
    =
    \int \frac{\parens{1 + t}^3}{t^{100}} \diff t
    =
    \int \frac{1 + 3t + 3t^2 + t^3}{t^{100}} \diff t  
    \\
    &=
    \int \parens{t^{-100} + 3t^{-99} + 3t^{-98} + t^{-97}} \diff t  
    =
    - \frac{1}{99 t^{99}} - \frac{3}{98 t^{98}} - \frac{3}{97 t^{97}} - \frac{1}{96 t^{96}} + c
    \\
    &=
    - \frac{1}{99 \parens{x - 1}^{99}} - \frac{3}{98 \parens{x - 1}^{98}} - \frac{3}{97 \parens{x - 1}^{97}} - \frac{1}{96 \parens{x - 1}^{96}} + c.
  \end{align*}
  \end{solution}

  \question[2] Spočtěte limitu funkce
  \begin{equation*}
    \lim_{x \to 0+} \parens{\ln \frac{1}{x}}^x.
  \end{equation*}
  
  \begin{solution}
    Platí
    \begin{equation*}
      \lim_{x \to 0+} x \ln \parens{\ln \frac{1}{x}}
      =
      \lim_{x \to 0+} \frac{\ln \parens{\ln \frac{1}{x}}}{\frac{1}{x}}
      \texteq{l'H}
      \lim_{x \to 0+} \frac{\frac{1}{\ln \frac{1}{x}} \cdot x \cdot \parens{-\frac{1}{x^2}}}{- \frac{1}{x^2}}
      =
      \lim_{x \to 0+} \frac{x}{\ln \frac{1}{x}}
      =
      \lim_{x \to 0+} x \, \cdot \lim_{x \to 0+} \frac{1}{\ln \frac{1}{x}} 
      =
      0 \cdot 0
      =
      0,
    \end{equation*} 
    kde jsme v druhé rovnosti využili l'Hôpitalova pravidla pro výpočet limity typu ``$\frac{\textrm{něco}}{\infty}$''. 
    
    Pro původní limitu potom dostáváme
    \begin{equation*}
      \lim_{x \to 0+} \parens{\ln \frac{1}{x}}^x
      =
      \exponential{\lim_{x \to 0+} x \ln \parens{\ln \frac{1}{x}}}
      =
      \exponential{0}
      =
      1.
    \end{equation*}
  \end{solution}
   
\end{questions}

\end{document}
