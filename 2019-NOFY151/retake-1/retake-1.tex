\documentclass[answers]{exam}

% encoding, language
\usepackage[T1]{fontenc}
\usepackage[czech]{babel}
\usepackage[utf8]{inputenc}

% geometry
\usepackage[a4paper]{geometry}
\geometry{
  a4paper,
  total={170mm,257mm},
  left=20mm,
  top=20mm}

% mathematics
\usepackage{amsmath}
\usepackage{amssymb}
\usepackage{amsthm}

% text appearence
\usepackage{libertine}
\usepackage{microtype}  % better general appearence of text

% graphics
\usepackage{graphicx}
\graphicspath{{figures/}}
\usepackage{tikz}
\usetikzlibrary{calc}
\usepackage{xcolor}

% colors
\definecolor{LightBlue}{HTML}{42bbed}
\definecolor{LightGray}{HTML}{616a6b}

% hypertext
\usepackage{hyperref}
\hypersetup{
	colorlinks=true,
	linkcolor=black,
	urlcolor=LightBlue
}

% bibliography
\usepackage{natbib}

% miscellaneous
\usepackage[parfill]{parskip}
\usepackage{nopageno}
\pagestyle{plain}
\usepackage{titling}

% exam class parameters
\bracketedpoints
\pointpoints{b}{b}
\renewcommand{\solutiontitle}{\noindent\textbf{Řešení: }}

% custom macros
\usepackage{../../macros}

% title
\title{\vspace{-3ex}Matematická analýza I (NOFY151) – 1. zápočtový test (opravný)\vspace{-1ex}}
\author{\vspace{-2ex}}
\date{\vspace{-2ex}29. října 2019}

\begin{document}
\maketitle

\begin{questions}
  \question[2] Dokažte pro $n \in \N$ a $x \in \R$ matematickou indukcí Moivreovu větu
  \begin{equation*}
    \parens{\cos x + \iunit \sin x}^n
    =
    \cos (nx) + \iunit \sin (nx).
  \end{equation*}
  
  \begin{solution}
    Pro $n = 1$ máme
    \begin{equation*}
      \cos x + \iunit \sin x
      =
      \cos x + \iunit \sin x,
    \end{equation*}
    a rovnost je tedy splněna. Ukažme, že pokud rovnost platí pro $n \in \N$ platí i pro $n + 1$. 
    \begin{align*}
      \parens{\cos x + \iunit \sin x}^{n + 1}
      &=
      \parens{\cos x + \iunit \sin x} \parens{\cos x + \iunit \sin x}^n
      \texteq{IP}
      \parens{\cos x + \iunit \sin x}
      \parens{\cos (nx) + \iunit \sin (nx)}
      \\
      &=
      \cos x \cos (nx) - \sin x \sin (nx)
      +
      \iunit
      \parens{ \sin x \cos (nx) + \cos x \sin (nx) }
      \\
      &=
      \cos \parens{(n + 1) x} + \iunit \sin \parens{(n + 1) x},
    \end{align*}
    kde jsme v poslední rovnosti využili goniometrické vzorce
    \begin{align*}
      \sin (x + y) &= \sin x \cos y + \cos x \sin y,
      \\
      \cos (x + y) &= \cos x \cos y - \sin x \sin y.
    \end{align*}
    
  \end{solution}
  
  \question[2] Ukažte, že pro $n \in \N$ platí nerovnost
  \begin{equation*}
    4^{n - 1}
    \ge
    n^2.
  \end{equation*}
  Pro důkaz použijte matematickou indukci.
  
  \begin{solution}
    Pro $n = 1, 2, 3$ pořadě dostaváme
    \begin{align*}
      1 &\ge 1,
      \\
      4 &\ge 4,
      \\
      16 &\ge 9,
    \end{align*}
    a pro tato $n$ je tedy nerovnost splněna. Ukažme, že pokud rovnost platí pro $n \ge 3$ platí i pro $n + 1$. Podle indukčního předpokladu máme
    \begin{equation*}
      4^n = 4 \cdot 4^{n - 1} \textge{IP} 4n^2.
    \end{equation*}
    Stačí tedy dokázat nerovnost
    \begin{equation*}
      4 n^2
      \ge
      (n + 1)^2,
    \end{equation*}
    která je ekvivalentní s
    \begin{equation*}
      2n^2 - 2n - 1 \ge 0.
    \end{equation*}
    Snadno zjistíme, že kořeny funkce $f(x) \defeq 2 x^2 - 2x - 1$ jsou $1 \pm \sqrt{3}$. Pro $x > 0$ je $f$ nezáporná na intervalu $[1 + \sqrt{3}, +\infty)$. Protože ale $1 + \sqrt{3} < 3 \le n$, poslední nerovnost je splněna.
   
  \end{solution}
  
  \question[2] Podmnožina reálných čísel $P$ je definována předpisem 
  \begin{equation*}
    P \defeq \set{(-1)^n \frac{n^2 + 1}{n^2 - 1}}{n \in \N \setminus \{1\}}.
  \end{equation*}
  Nalezněte její minimum, maximum, infimum a supremum (pokud existují). Ověřte z definice příslušných pojmů.
  
  \begin{solution}   
    \begin{itemize}
      \item $\min P = - \frac{5}{4}$: 
      
        Chceme ukázat, že pro $n \in \N \setminus \{1\}$ platí 
        \begin{equation*}
          - \frac{5}{4}
          \le
          (-1)^n \frac{n^2 + 1}{n^2 - 1}.
        \end{equation*}
        Pro sudá $n$ nerovnost zřejmě platí. Stačí tedy dokázat, že pro $n = \{3, 5, \dots\}$ platí
        \begin{equation*}
          - \frac{5}{4}
          \le
          -
          \frac{n^2 + 1}{n^2 - 1}.
        \end{equation*}
        Snadnou úpravou se ale ukáže, že to je ekvivalentní s $n \ge 3$.
      
      \item $\inf P = - \frac{5}{4}$:
      
        Protože existuje $\min P$, existuje i $\inf P$ a jejich hodnoty se rovnají.
      
      \item $\max P = \frac{5}{3}$:
      
        Chceme ukázat, že pro $n \in \N \setminus \{1\}$ platí 
        \begin{equation*}
          (-1)^n \frac{n^2 + 1}{n^2 - 1}
          \le
          \frac{5}{3}.
        \end{equation*}
        Pro $n = \{3, 5, \dots\}$ nerovnost zřejmě platí. Stačí tedy dokázat, že pro sudá $n$ platí
        \begin{equation*}
          \frac{n^2 + 1}{n^2 - 1}
          \le
          \frac{5}{3}.
        \end{equation*}
        Snadnou úpravou se ale ukáže, že to je ekvivalentní s $n \ge 2$.
        
      \item $\sup P = \frac{5}{3}$:
      
        Protože existuje $\max P$, existuje i $\sup P$ a jejich hodnoty se rovnají.

    \end{itemize}
  \end{solution}
  
  \question[2] Spočtěte (bez použití l'Hôpitalova pravidla nebo Taylorova rozvoje)
  \begin{equation*}
    \lim_{x \to 0} \parens{x + \exponential{x}}^{\frac{1}{x}}.
  \end{equation*}
  
  \begin{solution}
  Platí 
  \begin{align*}
    \lim_{x \to 0} \frac{1}{x} \ln \parens{x + \exponential{x}}
    &=
    \lim_{x \to 0} 
    \frac{1}{x} \ln \parens{x + \exponential{x}}
    \cdot
    \frac{x + \exponential{x} - 1}{x + \exponential{x} - 1}
    \\
    &=
    \lim_{x \to 0} \parens{1 + \frac{\exponential{x} - 1}{x}}
    \cdot
    \lim_{x \to 0} \frac{\ln \parens{x + \exponential{x}}}{x + \exponential{x} - 1}
    \\
    &=
    \parens{1 + 1} \cdot 1
    =
    2,
  \end{align*}
  kde v předposlední rovnosti jsme využili znalost limit
  \begin{equation*}
    \lim_{x \to 0} \frac{\exponential{x} - 1}{x} = 1,
    \qquad
    \lim_{y \to 0} \frac{\ln(1 + y)}{y} = 1,
  \end{equation*}
  a substituci $y = x + \exponential{x} - 1$.
  
  Konečně dostáváme
  \begin{equation*}
    \lim_{x \to 0} \parens{x + \exponential{x}}^{\frac{1}{x}}
    =
    \exponential{\lim_{x \to 0} \frac{1}{x} \ln \parens{x + \exponential{x}}}
    =
    \exponential{2}.
  \end{equation*}
  \end{solution}
  
  \question[2] Pro $n, m \in \N$ spočtěte (bez použití l'Hôpitalova pravidla nebo Taylorova rozvoje)
  \begin{equation*}
    \lim_{x \to 0} 
    \frac{
      \sqrt[n]{1 + x^2} - \sqrt[m]{1 - x^2}
    }{
      \sin(x^2)
    }.
  \end{equation*}
  
  \begin{solution}
    \begin{align*}
      \lim_{x \to 0} 
      \frac{
        \sqrt[n]{1 + x^2} - \sqrt[m]{1 - x^2}
      }{
        \sin(x^2)
      }
      &=
      \lim_{x \to 0}
      \frac{\sqrt[n]{1 + x^2} - \sqrt[m]{1 - x^2}}{x^2}
      \cdot
      \lim_{x \to 0}
      \frac{x^2}{\sin(x^2)} 
      \\
      &=
      \lim_{x \to 0}
      \parens{\frac{\sqrt[n]{1 + x^2} - 1}{x^2} - \frac{\sqrt[m]{1 - x^2} - 1}{x^2}}
      \cdot
      \lim_{x \to 0}
      \frac{x^2}{\sin(x^2)} 
      \\
      &=
      \parens{\frac{1}{n} + \frac{1}{m}} \cdot 1
      =
      \frac{m + n}{mn},
    \end{align*}
    kde v předposlední rovnosti jsme využili znalost limit
    \begin{equation*}
      \lim_{y \to 0} \frac{\sin y}{y} = 1,
      \qquad
      \lim_{y \to 0} \frac{\sqrt[n]{1 \pm y} - 1}{y} = \pm \frac{1}{n},
    \end{equation*}
    a substituci $y = x^2$.
    
  \end{solution}
   
\end{questions}

\end{document}
