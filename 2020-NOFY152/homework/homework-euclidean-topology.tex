\documentclass[answers]{exam}

% encoding, language
\usepackage[T1]{fontenc}
\usepackage[czech]{babel}
\usepackage[utf8]{inputenc}

% geometry
\usepackage[a4paper]{geometry}
\geometry{
  a4paper,
  total={170mm,257mm},
  left=20mm,
  top=20mm}

% mathematics
\usepackage{amsmath}
\usepackage{amssymb}
\usepackage{amsthm}

% text appearence
\usepackage{libertine}
\usepackage{microtype}  % better general appearence of text

% graphics
\usepackage{graphicx}
\graphicspath{{figures/}}
\usepackage{pgfplots}
\usepackage{xcolor}

% colors
\definecolor{LightBlue}{HTML}{42bbed}
\definecolor{LightGray}{HTML}{616a6b}
\definecolor{GraphColor}{HTML}{1d7ad1}

% hypertext
\usepackage{hyperref}
\hypersetup{
	colorlinks=true,
	linkcolor=black,
	urlcolor=LightBlue
}

% bibliography
\usepackage{natbib}

% miscellaneous
\usepackage[parfill]{parskip}
\usepackage{nopageno}
\pagestyle{plain}
\usepackage{titling}
\usepackage{enumitem}

% exam class parameters
\bracketedpoints
\pointpoints{b}{b}
\renewcommand{\solutiontitle}{\noindent\textbf{Řešení: }}

% custom macros
% correct spacing for \left \right commands
\let\originalleft\left
\let\originalright\right
\renewcommand{\left}{\mathopen{}\mathclose\bgroup\originalleft}
\renewcommand{\right}{\aftergroup\egroup\originalright}

% note/suggestion boxes
\newcommand*{\note}[1]{\smallskip\noindent\colorbox{SkyBlue}{
  \begin{minipage}{\linewidth}
  \textbf{Note }#1
  \end{minipage}}\smallskip
}
\newcommand*{\suggestion}[1]{\smallskip\noindent\colorbox{Dandelion}{
  \begin{minipage}{\linewidth}
  \textbf{Suggestion }#1
  \end{minipage}}\smallskip
}

% calculus (miscellaneous)
\newcommand*{\diff}{\mathrm{d}}
\newcommand*{\Diff}{\mathrm{D}}
\newcommand*{\loc}{\mathrm{loc}}
\newcommand*{\iunit}{\mathrm{i}}
\newcommand*{\inverse}[1]{#1^{-1}}
\newcommand*{\realpart}[1]{\mathrm{Re} \parens{#1}}
\newcommand*{\imagpart}[1]{\mathrm{Im} \parens{#1}}

% sets
\newcommand*{\set}[2]{\braces{\, #1 \mid #2 \,}}
\newcommand*{\N}{\ensuremath{{\mathbb{N}}}}
\newcommand*{\R}{\ensuremath{{\mathbb{R}}}}

% definition sign
\newcommand*{\defeq}{\mathrel{\overset{\makebox[0pt]{\mbox{\normalfont\tiny\sffamily def}}}{=}}}

% parens, brackets, braces, angles
\newcommand*{\parens}[1]{\left( #1 \right)}
\newcommand*{\brackets}[1]{\left[ #1 \right]}
\newcommand*{\braces}[1]{\left\{ #1 \right\}}
\newcommand*{\angles}[1]{\langle #1 \rangle}

% norms
\newcommand*{\abs}[2][]{\ensuremath{\left|#2\right|_{#1}}}
\newcommand*{\norm}[2][]{\ensuremath{\left\|#2\right\|_{#1}}}

% inner products
\newcommand*{\inner}[3][parens]{%
  \IfEqCase{#1}{%
    {parens}{\parens{#2, #3}}%
    {angles}{\angles{#2, #3}}%
  }[\PackageError{inner}{Undefined option to inner: #1}{}]%
}

% function spaces
\newcommand*{\lebs}[2]{L^{#1} \parens{#2}}
\newcommand*{\leblocs}[2]{L^{#1}_{\loc} \parens{#2}}


% title
\title{\vspace{-3ex}Matematická analýza II (NOFY152) – DÚ 7}
\author{Topologické pojmy v $\R^d$, funkce více proměnných}
\date{\vspace{-5ex}}

\begin{document}
\maketitle

\begin{questions}	
  \question Najděte vnitřek, uzávěr a hranici množiny $A \defeq M \setminus K$, kde
  \begin{align*}
  M &\defeq \set[|x| \le 1,\ 0 < y \le 2]{(x,y)\in \R^2},
  \\
  K &\defeq \set[x^2 + \parens{y-1}^2 = 1]{(x,y)\in \R^2}.
  \end{align*}
	
  \begin{solution}
		Platí
		\begin{align*}
		A^\circ&= \{(x,y)\in \R^2:\ |x|<1,\ 0<y<2\}\setminus K,\\
		\overline A&= \{(x,y)\in \R^2:\ |x|\le 1,\ 0\le y\le2\},\\
		\partial A&=\{(\pm1,y)\in \R^2:\ 0\le y\le2\}\cup\{(x,0)\in \R^2:\ |x|\le1\}\cup\{(x,2)\in \R^2:\ |x|\le1\}\cup K.
		\end{align*}
		Pozorujeme, že $A^\circ \cup \partial A = \overline{A}$.
  \end{solution}
  
  \question
  Pro kter\'e hodnoty parametru $\alpha\in\R$ plat\'i, \v ze po\v c\'atek le\v z\'i v uz\'av\v eru mno\v ziny
  \begin{equation*}
  	M
  	\defeq
  	\set[k\in\mathbb Z\setminus\{0\}]{(k^\alpha\cos k,k^\alpha\sin k)}?
  \end{equation*}
  Zdůvodněte.
  
  \begin{solution}
		Nejprve p\v ripome\v nme definici. 
		\begin{enumerate}
		 \item \label{1} Je-li $\alpha=\frac{p}{q}$ racion\'aln\'i, kde $p,q$ jsou nesoud\v eln\'e a $q$ lich\'e, pak $k^\alpha=k^{\frac{p}{q}}$ je definov\'ano pro ka\v zd\'e $k\in\mathbb Z\setminus\{0\}$,
		 \item Nen\'i-li $\alpha$ tvaru jako v \ref{1}, pak  $k^{\alpha}=e^{\alpha\ln k}$ je definov\'ano jen pro $k>0,\ k\in\mathbb Z$. V tomto p\v r\'ipad\v e  m\r u\v zeme uva\v zovat jen $k>0$. 
		\end{enumerate}
		
		M\'ame naj\'it v\v sechna $\alpha$ tak, aby pro ka\v zd\'e $\varepsilon>0$ platilo $U_\varepsilon((0,0))\cap M\ne\emptyset$. Zde 
		$$U_{\varepsilon}((0,0))=\{(x,y)\in\mathbb R^2:\ \|(x,y)\|=\sqrt{x^2+y^2}<\varepsilon\},$$
		je otev\v ren\'a koule o polom\v eru $\varepsilon>0$ se st\v redem v po\v c\'atku. Plat\'i 
		$$\|(k^\alpha\cos k,k^\alpha\sin k)\|=|k^\alpha|=|k|^{\alpha}.$$
		Je-li $\alpha\ge0$, pak $|k|^\alpha\ge1$ a tedy $U_\frac{1}{2}((0,0))\cap M=\emptyset$. Je-li naopak $\alpha<0$, pak $|k|^\alpha$ kles\'a pro $k\to+\infty$ monotonn\v e k nule, existuje tedy $k_0\in\mathbb N$ tak, \v ze pro $k\ge k_0$ plat\'i $|k|^\alpha<\varepsilon$. Tedy $U_\varepsilon((0,0))\cap M\ne\emptyset$ pro ka\v zd\'e $\varepsilon>0$.
		
		Alternativně lze argumentovat přes hromadné body, což jsou v tomto případě body, které dostaneme pro $k \to +\infty$ nebo $k \to -\infty$. Podobné úvahy jako výše pak vedou k závěru, že $\parens{0, 0}$ je hromadný bod dané množiny, jen pokud $\alpha < 0$. 
		
		Z\'av\v er: Bod $(0,0)$ le\v z\'i v uz\'av\v eru mno\v ziny $M$ jen pro $\alpha<0$.
		
  \end{solution}
  
  \question
  Je množina 
  \begin{equation*}
  	M = \set[x^2+y^2-z^2>1,\ |x|\le1,\ |y|\le1]{(x,y,z)\in \R^3}
  \end{equation*}
  omezená? Zdůvodněte.
	
	\begin{solution}
		Z definice $M$ plyne omezen\'i $x^2+y^2>z^2+1$. D\'ale pro $(x,y,z)\in M$ plat\'i  
		\begin{align*}
		|x|+|y|\le2,\qquad 4=2^2\ge (|x|+|y|)^2=|x|^2+|y|^2+2|x||y|\ge x^2+y^2>z^2+1.
		\end{align*}
		Tud\'i\v z $z^2<3$ a $|z|<\sqrt3$. Mno\v zina $M$ le\v z\'i uvnit\v r omezen\'e mno\v ziny 
		$$\{(x,y,z):\ |z|<\sqrt3,\ |x|\le1,\ |y|\le1\},$$
		která je podmnožinou krychle se středem v počátku a délkou strany $2\sqrt{3}$, tj. 
		\begin{equation*}
			M \subset B_{\sqrt{3}}\parens{0} = \set[{\abs{\vec{x}}}_{\infty} < \sqrt{3}]{\vec{x} = \parens{x, y, z} \in \R^3}.
		\end{equation*}
		$M$ je tedy omezená.
	\end{solution}
	
	\question
	Nechť $A,B\subset \R^n$. Rozhodn\v ete, zda plat\'i n\'asleduj\'ic\'i rovnosti nebo alespo\v n jedna inkluze, tj. doka\v zte nebo najd\v ete protip\v r\'iklady,
	\begin{enumerate}[label=(\roman*)]
		\item $\overline{\partial A}=\partial \overline{A}$
		\item $\partial (A\cup B)=\partial A\cup\partial B$
		\item $\overline{A\cap B}=\overline A\cap\overline B$
	\end{enumerate}
	
	\begin{solution}
		\begin{enumerate}[label=(\roman*)]
			\item 
				Inkluze $\overline{\partial A}\subseteq{\partial \overline A}$ neplat\'i. Sta\v c\'i vz\'it $A=\Q$. Pak $\overline{\partial A}=\overline\R=\R$ a $\partial\overline\Q=\partial\R=\emptyset$.
				
				Zkusme nyn\'i $\supseteq$. Je-li $x\in\partial\overline A$ a $\varepsilon>0$, pak $U_\varepsilon(x)\cap\overline A\ne\emptyset$ a $U_\varepsilon(x)\cap(\R^n\setminus\overline A)\ne\emptyset$. Jeliko\v z $(\R^n\setminus{\overline A})\subseteq (\R^n\setminus A)$, pak i $U_\varepsilon(x)\cap (\R^n\setminus A)\ne\emptyset$. Zvolme $x'\in U_\varepsilon(x)\cap\overline A$. Pak $U_\varepsilon(x)$ je otev\v ren\'e okol\'i $x'$ a jeliko\v z $x'\in\overline A$, pak $U_\varepsilon(x)$ mus\'i obsahovat bod z $A$. Tedy pro ka\v zd\'e $\varepsilon>0$ plat\'i $U_\varepsilon(x)\cap A\ne\emptyset$. Ka\v zd\'e otev\v ren\'e okol\'i bodu $x$ tedy protne $A$ i $\R^n\setminus A$. To ale z definice znamen\'a, \v ze $x\in\partial A$ a tedy i $x\in\overline{\partial A}$. Dok\'azali jsme $\overline{\partial A}\supseteq{\partial \overline A}$.
				
			\item 
				Inkluze $\partial (A\cup B)\supseteq\partial A\cup\partial B$ neplat\'i. Sta\v c\'i vz\'it otev\v ren\'e intervaly $A=(-1,1)$ a $B=(0,2)$. Pak $\partial({A\cup B})=\partial(-1,2)=\{-1,2\}$ a $\partial A\cup\partial B=\{-1,1\}\cup\{0,2\}=\{-1,1,0,2\}$.
				
				Obr\'acen\'a inkluze $\partial (A\cup B)\subseteq\partial A\cup\partial B$ plat\'i a nyn\'i ji dok\'a\v zeme nep\v r\'imo. Nen\'i-li $x\in\partial A\cup\partial B$, pak $x\not\in\partial A$ a $x\not\in\partial B$. Pak ale $x$ je vnit\v rn\'i bod $A$ nebo $\R^n\setminus A$ a sou\v casn\v e je to vnit\v rn\'i bod $B$ nebo $\R^n\setminus B$. Je-li $x\in A^o$ nebo $x\in B^o$, pak m\'ame vyhr\'ano, nebo\v t pak je $x$ nutn\v e i vnit\v rn\'im bodem $A\cup B$. Zb\'yv\'a tedy probrat p\v r\'ipad, kde $x$ je vnit\v rn\'im bodem $\R^n\setminus A$ i $\R^n\setminus B$. Najdeme $\varepsilon_1,\varepsilon_2$ tak, aby $U_{\varepsilon_1}(x)\subseteq(\R^n\setminus A) $ a $U_{\varepsilon_2}(x)\subseteq(\R^n\setminus B)$. Pak ale $U_\varepsilon(x)\subset \R^n\setminus(A\cup B)$, kde $\varepsilon=\min\{\varepsilon_1,\varepsilon_2\}$. Tud\'i\v z $x$ je vnit\v rn\'i bodem $\R^n\setminus(A\cup B)$ a nem\r u\v ze tedy le\v zet na hranici t\'eto mno\v ziny. Uk\'azali jsme implikaci $x\not\in\partial A\cup\partial B\Rightarrow x\not\in \partial (A\cup B)$ a t\'im i $x\in(\partial A\cup\partial B)\Rightarrow x\in\partial A\cup\partial B$. D\r ukaz je hotov.
				
			\item 
				Inkluze $\overline{A\cap B}\supseteq\overline A\cap\overline B$ neplat\'i. Sta\v c\'i vz\'it otev\v ren\'e intervaly $A=(-1,0)$ a $B=(0,1)$. Pak $\overline{A\cap B}=\overline\emptyset=\emptyset$ a sou\v casn\v e $\overline{A}\cap\overline B=[-1,0]\cap[0,1]=\{0\}$.
			
				Dok\'a\v zeme $\subseteq$. Je-li $x\in\overline {A \cap B}$, pak pro ka\v zd\'e $\varepsilon>0$ plat\'i $U_\varepsilon(x)\cap (A \cap B)\ne\emptyset$. Speci\'aln\v e $U_\varepsilon(x)\cap A\ne\emptyset$ a $U_\varepsilon(x)\cap B\ne\emptyset$.   Uk\'azali jsme, \v ze libovoln\'e otev\v ren\'e okol\'i $x$ m\'a netrivi\'aln\'i pr\r unik s $A$ i s $B$. Tedy $x\in\overline A$ a $x\in\overline B$, nutn\v e tedy $x\in\overline A\cap\overline B$.
		\end{enumerate}
	\end{solution}
	
	\question
	Spo\v ct\v ete limity (pokud existuj\'i)
	\begin{enumerate}[label=(\roman*)]
		\item $\lim\limits_{(x,y)\to(0,0)}\frac{x^2-y^2+xy}{|x|+|y|}$, 
		\item $\lim\limits_{(x,y)\to(0,0)}(\cos(x+y))^{\frac{1}{x^2+y^2}}$.
	\end{enumerate}
	
	\begin{solution}
	\begin{enumerate}[label=(\roman*)]
		\item \label{i}
			Pomocí Youngovy nerovnosti dostáváme
			\begin{equation*}
				0
				\le
				\abs{\frac{x^2 - y^2 + xy}{\abs{x} + \abs{y}}}
				\le
				\frac{x^2 + y^2 + \abs{xy}}{\abs{x} + \abs{y}}
				\le
				\frac{3}{2} \frac{x^2 + y^2}{\abs{x} + \abs{y}}
				\le
				\frac{3}{2} \sqrt{x^2 + y^2}
				=
				\frac{3}{2} {\abs{\parens{x, y}}}_{\mathrm{2}},
			\end{equation*}
			odkud vidíme, že zadaná limita je rovna 0. V poslední nerovnosti jsme navíc využili
			\begin{equation*}
				\frac{\sqrt{x^2 + y^2}}{\abs{x} + \abs{y}}
				\le
				1
				\quad \iff \quad
				x^2 + y^2
				\le
				\parens{\abs{x} + \abs{y}}^2
				\quad \iff \quad
				0
				\le
				\abs{xy}.
			\end{equation*}
			
			Alternativně můžeme postupovat takto
			\begin{equation*}
				0
				\le
				\abs{\frac{x^2 - y^2 + xy}{\abs{x} + \abs{y}}}
				\le
				\frac{x^2 + y^2 + \abs{xy}}{\abs{x} + \abs{y}}
				\le
				\frac{\abs{x}^2 + \abs{y}^2 + 2 \abs{xy}}{\abs{x} + \abs{y}}
				=
				\frac{\parens{\abs{x} + \abs{y}}^2}{\abs{x} + \abs{y}}
				=
				\abs{x} + \abs{y}
				=
				{\abs{\parens{x, y}}}_{\mathrm{1}}.
			\end{equation*}			
		\item \label{ii}
			Ukažme, že limita neexistuje. Na paprsku $y = x$ dostáváme
			\begin{equation*}
				\lim_{x \to 0} \parens{\cos 2x}^{\frac{1}{2 x^2}}
				=
				\lim_{x \to 0} \exponential{\frac{\ln \parens{\cos 2x}}{2 x^2}}
				=
				\exponential{\lim_{x \to 0} \frac{\ln \parens{\cos 2x}}{2 x^2}}
				=
				\exponential{\lim_{x \to 0} \frac{\cos 2x - 1}{2x^2}}
				=
				\exponential{-1},
			\end{equation*}
			kde jsme využili věty o limitě složené funkce a limitě součinu a znalost základních limit
			\begin{equation*}
				\lim_{x \to 0} \frac{\ln \parens{1 + x}}{x} = 0,
				\qquad
				\lim_{x \to 0} \frac{1 - \cos x}{x} = \frac{1}{2}.
			\end{equation*}
			Na druhou stranu na paprsku $y = -x$ máme
			\begin{equation*}
				\lim_{x \to 0} \parens{\cos 0}^{\frac{1}{2 x^2}}
				=
				\lim_{x \to 0} 1^{\frac{1}{2 x^2}}
				=
				1.
			\end{equation*}
			Protože se limity neshodují, zadaná limita neexistuje.
	\end{enumerate}	
	\end{solution}
	
	\question
	Jestli\v ze dodefinujeme funkce z p\v redchoz\'iho p\v r\'ikladu v po\v c\'atku nulou, rozhodn\v ete, zda jsou takto definované funkce spojit\'e nebo omezen\'e na n\v ejak\'em okol\'i po\v c\'atku.
	
	\begin{solution}
		\begin{enumerate}[label=(\roman*)]
			\item
				Podle Příkladu 5\ref{i} víme, že funkce
				\begin{equation*}
					f\parens{x,y}
					\defeq
					\begin{cases}
						\frac{x^2 - y^2 + xy}{\abs{x} + \abs{y}} &(x, y) \neq (0, 0),
						\\
						0 &(x, y) = \parens{0, 0},
					\end{cases}
				\end{equation*}
				je spojitá v počátku, a tudíž je i omezená na nějakém okolí počátku (podle věty o nabývání extrémů pro spojitou funkci na kompaktní množině).
				
			\item 
				Podle Příkladu 5\ref{ii} není funkce
				\begin{equation*}
					f\parens{x,y}
					\defeq
					\begin{cases}
						(\cos(x+y))^{\frac{1}{x^2+y^2}} &(x, y) \neq (0, 0),
						\\
						0 &(x, y) = \parens{0, 0},
					\end{cases}
				\end{equation*}
				spojitá v počátku. V dalším uvažujme funkci $f$ na množině
				\begin{equation*}
					M \defeq \set[\abs{x + y} < \frac{\pi}{2}]{\parens{x, y} \in \R^2},
				\end{equation*}
				neboť na takové množině je $f$ definovaná a zároveň obsahuje počátek. Pro $\parens{0, 0} \neq \parens{x, y} \in M$ je z definice funkce $f$ omezená zdola, neboť
				\begin{equation*}
					(\cos(x+y))^{\frac{1}{x^2+y^2}}
					=
					\exponential{\frac{\ln \parens{\cos(x+y)}}{x^2+y^2}}
					>
					0.
				\end{equation*}
				Dále platí
				\begin{equation*}
					\exponential{\frac{\ln \parens{\cos(x+y)}}{x^2+y^2}} \le 1
					\quad \iff \quad
					\frac{\ln \parens{\cos(x+y)}}{x^2+y^2} \le 0
					\quad \iff \quad
					\cos \parens{x + y} \le 1,
				\end{equation*}
				což je splněno pro všechna $\parens{x, y} \in M$, a tedy $f$ je na této množině omezená i shora. Existuje tedy okolí počátku, na kterém je funkce $f$ omezená.
		\end{enumerate}		
	\end{solution}	
	
  \question
	Určete definiční obor n\'asleduj\'ic\'ich funkc\'i a spo\v ct\v ete jejich první parci\'aln\'i derivace
	\begin{enumerate}[label=(\roman*)]
		\item $f(x,y,z)=\cos (x+y)\sin (x-y+z)$,
		\item $f(x,y)=\exponential{{\operatorname{tg} \parens{xy}}}$.
	\end{enumerate}
	    
  \begin{solution}
	  \begin{enumerate}[label=(\roman*)]
	  \item 
	  	Zřejmě platí $D_f = \R^3$. S využitím součtového vzorce $\cos\parens{x \pm y} = \cos x \cos y \mp \sin x \sin y$, potom pro první parciální derivace na celém $D_f$ dostáváme
	  	\begin{align*}
	  		\pd{f}{x}\parens{x,y,z}
	  		&=
	  		-\sin\parens{x+y}\sin\parens{x-y+z} + \cos\parens{x+y}\cos\parens{x-y+z}
	  		=
	  		\cos\parens{2x+z},
	  		\\
	  		\pd{f}{y}\parens{x,y,z}
	  		&=
	  		-\sin\parens{x+y}\sin\parens{x-y+z} - \cos\parens{x+y}\cos\parens{x-y+z}
	  		=
	  		-\cos\parens{2y-z},
	  		\\
	  		\pd{f}{z}\parens{x,y,z}
	  		&=
	  		\cos\parens{x+y}\cos\parens{x-y+z}.
	  	\end{align*}
	  
	  \item 
	  	Z podmínky $xy \neq \frac{\parens{2k+1}\pi}{2}$ dostáváme
	  	\begin{equation*}
	  		D_f = \R^2 \setminus \set[x \in \R \setminus \set{0}, k \in \Z]{\parens{x, \frac{\parens{2k+1}\pi}{2x}}}.
	  	\end{equation*}
	  	Na $D_f$ potom máme
	  	\begin{align*}
	  		\pd{f}{x}\parens{x,y}
	  		&=
	  		\frac{y \exponential{{\operatorname{tg} \parens{xy}}}}{\cos^2 \parens{xy}},
	  		\\
	  		\pd{f}{y}\parens{x,y}
	  		&=
	  		\frac{x \exponential{{\operatorname{tg} \parens{xy}}}}{\cos^2 \parens{xy}}.
	  	\end{align*}
		\end{enumerate}
  \end{solution}
  
 	\question
 	Najd\v ete směrovou derivaci funkce 
 	\begin{equation*}
 	f(x,y)
 	=
 	\begin{cases}
 		\frac{x^4+2x^2y+y^3}{x^2+y^2} &(x,y)\ne \parens{0, 0},
 		\\
 		0 &\parens{x,y}=\parens{0,0},
 	\end{cases}
 	\end{equation*}
 	v po\v c\'atku v obecn\'em sm\v eru $(u,v)\in\R^2$, $u^2 + v^2 = 1$.
 	
 	\begin{solution}
 	Označme si $\vec{v} \defeq \parens{u, v}$. Podle definice derivace ve směru máme
 	\begin{equation*}
 		\pd{f}{\vec{v}}\parens{\parens{0, 0}}
 		=
 		\lim_{h \to 0} \frac{f\parens{\parens{0,0} + h \parens{u, v}} - f\parens{\parens{0,0}}}{h}
 		=
 		\lim_{h \to 0} \frac{f\parens{\parens{hu, hv}}}{h}
 		=
 		\lim_{h \to 0} \frac{\frac{h^4 u^4 + 2h^3 u^2 v + h^3 v^3}{h^2 u^2 + h^2 v^2}}{h}.
 	\end{equation*}
 	S využitím $u^2 + v^2 = 1$, pak dostáváme
 	\begin{equation*}
 		\pd{f}{\vec{v}}\parens{\parens{0, 0}}
 		=
 		\lim_{h \to 0}
 		\parens{h u^4 + 2 u^2 v + v^3}
 		=
 		2 u^2 v + v^3.
 	\end{equation*}
 	\end{solution}
  
\end{questions}

\end{document}
