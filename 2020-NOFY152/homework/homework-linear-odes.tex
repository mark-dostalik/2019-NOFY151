\documentclass[answers]{exam}

% encoding, language
\usepackage[T1]{fontenc}
\usepackage[czech]{babel}
\usepackage[utf8]{inputenc}

% geometry
\usepackage[a4paper]{geometry}
\geometry{
  a4paper,
  total={170mm,257mm},
  left=20mm,
  top=20mm}

% mathematics
\usepackage{amsmath}
\usepackage{amssymb}
\usepackage{amsthm}

% text appearence
\usepackage{libertine}
\usepackage{microtype}  % better general appearence of text

% graphics
\usepackage{graphicx}
\graphicspath{{figures/}}
\usepackage{pgfplots}
\usepackage{xcolor}

% colors
\definecolor{LightBlue}{HTML}{42bbed}
\definecolor{LightGray}{HTML}{616a6b}
\definecolor{GraphColor}{HTML}{1d7ad1}

% hypertext
\usepackage{hyperref}
\hypersetup{
	colorlinks=true,
	linkcolor=black,
	urlcolor=LightBlue
}

% bibliography
\usepackage{natbib}

% miscellaneous
\usepackage[parfill]{parskip}
\usepackage{nopageno}
\pagestyle{plain}
\usepackage{titling}
\usepackage{enumitem}

% exam class parameters
\bracketedpoints
\pointpoints{b}{b}
\renewcommand{\solutiontitle}{\noindent\textbf{Řešení: }}

% custom macros
/home/mark/disertation/github/macros.sty

% title
\title{\vspace{-3ex}Matematická analýza II (NOFY152) – DÚ 5}
\author{Integrační faktor, lineární ODR s konstantními koeficienty}
\date{\vspace{-5ex}}

\begin{document}
\maketitle

Nalezněte maximální řešení rovnic

\begin{questions}	
	\question
	\begin{equation*}
	(1-x^2)y'+xy=1,\quad y(0)=1
	\end{equation*}
	
	\begin{solution}
		Nejprve p\v repi\v sme rovnici do tvaru
		\begin{equation}\label{ODR 1a}
		y'+\frac{xy}{1-x^2}=\frac{1}{1-x^2},\quad x\ne\pm1.
		\end{equation}
		V dalším se omezíme na $x \in \parens{-1, 1}$, neboť na tomto intervalu se nachazí počáteční podmínka. Máme
		\begin{align*}
		\int\frac{x}{1-x^2}\, \diff x=-\frac{1}{2}\ln|1-x^2|+c=\ln\frac{1}{\sqrt{1-x^2}}+c.
		\end{align*}
		Rovnici (\ref{ODR 1a}) p\v ren\'asob\'ime funkc\'i $e^{\ln\frac{1}{\sqrt{1-x^2}}}=\frac{1}{\sqrt{1-x^2}}$ a dostaneme
		\begin{equation}
		\label{eq:9}
		\left(y\frac{1}{\sqrt{1-x^2}}\right)'=\frac{1}{\sqrt{1-x^2}}y'+\frac{xy}{(1-x^2)^{\frac{3}{2}}}=\frac{1}{(1-x^2)^{\frac{3}{2}}}. 
		\end{equation}
		Protože platí
		\begin{align*}
			\int \! 
				\frac{1}{\sqrt{1 - x^2}} 
			\, \diff x
			&=
	    \left| 
	      \begin{aligned}
	        u &= \frac{1}{\sqrt{1-x^2}} & v &= x
	        \\
	        u' &= \frac{x}{\parens{1 - x^2}^{\frac{3}{2}}} & v' &= 1
	      \end{aligned}
	    \right|
	    =
	    \frac{x}{\sqrt{1 - x^2}}
	    -
	    \int \!
		    \frac{x^2}{\parens{1 - x^2}^{\frac{3}{2}}}
		  \, \diff x
		  \\
	    &=
	    \frac{x}{\sqrt{1 - x^2}}
	    +
	    \int \! 
	    	\frac{1}{\sqrt{1 - x^2}} 
	    \, \diff x
	    -
	    \int \! 
	    	\frac{1}{\parens{1 - x^2}^{\frac{3}{2}}} 
	   	\, \diff x,
		\end{align*}
		dostáváme odsud, že
		\begin{equation*}
		\int\frac{1}{(1-x^2)^{\frac{3}{2}}}\,\diff x
		=
		\frac{x}{\sqrt{1-x^2}}+d,
		\end{equation*}
		a tud\'i\v z 
		\begin{align*}
		y(x)&=\sqrt{1-x^2}\big(\frac{x}{\sqrt{1-x^2}}+d\big)=d\sqrt{1-x^2}+x. 
		\end{align*}
		Jeliko\v z $1=y(0)=d$, pak funkce
		\begin{equation}\label{reseni1}
		y(x)=\sqrt{1-x^2}+x,\ x\in(-1,1)
		\end{equation}
		\v re\v s\'i zadanou počáteční úlohu. Jeliko\v z $\lim_{x \to -1+}y'(x)=+\infty$ a $\lim_{x \to 1-} y'(x)=-\infty$, pak (\ref{reseni1}) nelze roz\v s\'i\v rit jako $\mathcal C^1$ funkci na \v z\'adn\'y v\v et\v s\'i otevřený interval. Tud\'i\v z (\ref{reseni1}) je maxim\'aln\'i \v re\v sen\'i.
		
		Alternativně můžeme rovnici \eqref{eq:9} přímo integrovat od $0$ do $x$ s využitím počáteční podmínky
		\begin{equation*}
			\frac{y(x)}{\sqrt{1 - x^2}} - y(0)
			=
			\int_0^x \! \frac{1}{\parens{1 - s^2}^{\frac{3}{2}}} \, \diff s
			=
			\brackets{\frac{s}{\sqrt{1 - s^2}}}_0^x
			=
			\frac{x}{\sqrt{1 - x^2}},
		\end{equation*}
		což implikuje \eqref{reseni1}.
	\end{solution}
	
	\question
	\begin{equation*}
		x(x-1)y'=y-1
	\end{equation*}
	
	\begin{solution}
		(Lze zav\'est novou funkci $z=y-1$, pak se rovnice zjednodu\v s\v s\'i na $x(x-1)z'=z$.) 
		
		Ze zadání je zřejmé, že $y(x) = 1$ je stacionárním řešením úlohy pro $x \in \R$. Pro nalezení netriviálních řešení nejprve p\v repi\v sme rovnici do tvaru
		\begin{equation}\label{ODR 2a}
		y'-\frac{y}{x(x-1)}=-\frac{1}{x(x-1)},
		\end{equation}
		kterou řešíme na intervalech, kde $x \neq 0, 1$. M\'ame
		\begin{align*}
		-\int\frac{1}{x(x-1)} \, \diff x&=-\int\bigg(\frac{-1}{x}+\frac{1}{x-1}\bigg) \, \diff x \\
		&=\int\bigg(\frac{1}{x}-\frac{1}{x-1}\bigg)\, \diff x =\ln|x|-\ln|x-1|+c=\ln\left|\frac{x}{x-1}\right|+c. 
		\end{align*}
		Nyn\'i rozli\v sujme dva p\v r\'ipady, a to 
		\begin{enumerate}[label=(\alph*)]
		 \item  $x\in(-\infty,0)\cup(1,+\infty)$, pak $\frac{x}{x-1}>0$, a 
		 \item  $x\in(0,1)$, kde $\frac{x}{x-1}<0$.
		\end{enumerate}
		
		Jestli\v ze plat\'i (a), pak rovnici (\ref{ODR 2a}) p\v ren\'asob\'ime funkc\'i $e^{\ln\frac{x}{x-1}}=\frac{x}{x-1}$ a dostaneme
		\begin{align*}
		\left(y\frac{x}{x-1}\right)'=\frac{x}{x-1}y'-\frac{y}{(x-1)^2}=-\frac{1}{(x-1)^2}. 
		\end{align*}
		D\'ale
		\begin{align*}
		-\int\frac{1}{(x-1)^2} \, \diff x&=\frac{1}{x-1}+d,
		\end{align*}
		a tud\'i\v z 
		\begin{align*}
		y(x)&=\frac{x-1}{x}\left(\frac{1}{x-1}+d\right)=d\frac{x-1}{x}+\frac{1}{x}=d+\frac{1-d}{x},\quad d\in\mathbb R. 
		\end{align*}
		
		Jestli\v ze plat\'i (b), pak rovnici (\ref{ODR 2a}) p\v ren\'asob\'ime funkc\'i $e^{\ln\frac{-x}{x-1}}=\frac{-x}{x-1}$. Pak ale dostaneme rovnici, kter\'a se od t\'e v (a) li\v s\'i jen o znam\'enko, jedn\'a se tud\'i\v z o stejnou diferenci\'aln\'i rovnici. \v Re\v sen\'i (\ref{ODR 2a}) je tedy na intervalech $\parens{-\infty, 0}$, $\parens{0, 1}$ a $\parens{1, +\infty}$ dáno předpisem
		\begin{equation}\label{reseni2}
		y(x)=d+\frac{1-d}{x},\quad d\in\mathbb R.
		\end{equation}
		Řešení (\ref{reseni2}) se dají napojit v $x=1$, je-li nav\'ic $d=1$, pak dokonce i v $x=0$. Maxim\'aln\'i \v re\v sen\'i zadané rovnice jsou tedy 
		\begin{align*}\label{reseni2a}
		y(x)&=d+\frac{1-d}{x},\quad d\in\mathbb R\setminus\{1\},\, x\in \parens{-\infty,0} \lor x \in \parens{0, + \infty} ,\\
		y(x)&=1,\quad x\in\mathbb R.
		\end{align*}		
	\end{solution}
	
  \question
  \begin{equation*}
    y'' - \frac{1 + \ln^2 2}{\ln 2}y' + y 
    = 
    2^x
  \end{equation*}
    
  \begin{solution}
  	Charakteristický polynom má tvar
  	\begin{equation*}
  		\lambda^2 - \frac{1 + \ln^2 2}{\ln 2} \lambda + 1
  		=
  		\parens{\lambda - \ln 2} \parens{\lambda - \frac{1}{\ln 2}}.
  	\end{equation*}
  	Jednonásobné kořeny $\ln 2$, $\frac{1}{\ln 2}$ dávají fundamentální systém
  	\begin{equation*}
  		\set{\exponential{\parens{\ln 2}x}, \exponential{\frac{x}{\ln 2}}}
  		=
  		\set{2^x, \exponential{\frac{x}{\ln 2}}}.
  	\end{equation*}
  	
  	Pravou stranu zadané rovnice si přepíšeme do tvaru $\exponential{x \ln 2}$, což už má tvar speciální pravé strany. Protože $\ln 2$ je jednonásobný kořen charakteristického polynomu, partikulární řešení $y_p$ hledáme ve tvaru
  	\begin{equation*}
  		y_p(x)
  		=
  		A x \exponential{\parens{\ln 2} x}
  		=
  		A x 2^x.
  	\end{equation*}
  	První a druhá derivace partikulárního řešení vychází
  	\begin{align*}
  		y_p'(x) &= A 2^x + A \parens{\ln 2} x 2^x,
  		\\
  		y_p''(x) &= 2 A \parens{\ln 2} 2^x + A \parens{\ln^2 2} x 2^x.
  	\end{align*}
  	Po dosazení do zadané rovnice pak dostáváme, že musí platit
  	\begin{equation*}
  		A = \frac{\ln 2}{\ln^2 2 - 1}.
  	\end{equation*}
  	Obecné řešení definované pro $x \in \R$ je tedy
  	\begin{equation*}
  		y(x) = C_1 2^x + C_2 \exponential{\frac{x}{\ln 2}} + \frac{\ln 2}{\ln^2 2 - 1} x 2^x,
  	\end{equation*}
  	kde $C_1, C_2 \in \R$.
  \end{solution}

  \question
  \begin{equation*}
    y''' - y'' + 4y' - 4y
    =
    40 \cos^2 x
  \end{equation*}
  
  \begin{solution}
  	Charakteristický polynom $p$ má tvar
  	\begin{equation*}
  		p(\lambda) = \lambda^3 - \lambda^2 + 4 \lambda - 4.
  	\end{equation*}
  	Uhádneme, že jeden z kořenů je $1$. Dělením mnohočlenu mnohočlenem pak dostaneme
  	\begin{equation*}
  		p(\lambda) = \parens{\lambda - 1} \parens{\lambda^2 + 4}.
  	\end{equation*}
  	Jednonásobné kořeny $1$, $\pm 2\iunit$ dávají fundamentální systém
  	\begin{equation*}
  		\set{\exponential{x}, \cos 2x, \sin 2x}.
  	\end{equation*}
  	
  	Pravou stranu zadané rovnice si přepíšeme pomocí identity $\cos^2 x = \frac{1 + \cos 2x}{2}$ do tvaru
  	\begin{equation*}
  		20 + 20 \cos 2x.
  	\end{equation*}
  	Pravá strana je tedy součtem dvou funkcí, které mají tvar speciální pravé strany, a partikulární řešení budeme tudíž hledat jako součet partikulárních příslušejících jednotlivým sčítancům. Partikulární řešení $y_{p, 1}$ odpovídající pravé straně $20$ se dá jednoduše uhádnout
  	\begin{equation*}
  		y_{p, 1}(x) = -5.
  	\end{equation*}
  	Protože $2\iunit$ je jednonásobný kořen charakteristického polynomu, partikulární řešení $y_{p, 2}$ odpovídající pravé straně $20 \cos 2x$ budeme hledat ve tvaru
  	\begin{equation*}
  	 y_{p, 2}(x)
  	 =
  	 x \parens{A \cos 2x + B \sin 2x}.
  	\end{equation*}
  	První, druhá a třetí derivace $y_{p, 2}$ vychází
  	\begin{align*}
  		y_{p, 2}'(x) &= A \cos 2x + B \sin 2x + x \parens{-2 A \sin 2x + 2 B \cos 2x},
  		\\
  		y_{p, 2}''(x) &= - 4 A \sin 2x + 4 B \cos 2x + x \parens{- 4 A \cos 2x - 4 B \sin 2x},
  		\\
  		y_{p, 2}'''(x) &= - 12 A \cos 2x - 12 B \sin 2x + x \parens{8 A \sin 2x - 8 B \cos 2x}.
   	\end{align*}
  	Po dosazení do zadané rovnice pak dostáváme, že musí platit
  	\begin{equation*}
  		A = -2, \quad B = -1.
  	\end{equation*}
  	Obecné řešení definované pro $x \in \R$ je tedy
  	\begin{equation*}
  		y(x) = C_1 \exponential{x} + C_2 \cos 2x + C_3 \sin 2x - x \parens{2 \cos 2x + \sin 2x} - 5, 
  	\end{equation*}
  	kde $C_1, C_2, C_3 \in \R$.
  \end{solution}

  \question
  \begin{equation*}
    y'' + 2y' + 2y = \frac{\exponential{-x}}{\sin x}
  \end{equation*}
  
  \begin{solution}
  	Kořeny charakteristického polynomu
  	\begin{equation*}
  		\lambda^2 + 2 \lambda + 2
  	\end{equation*}
  	jsou $1 \pm \iunit$. Fundamentální systém je tedy
  	\begin{equation*}
	  	\set{\exponential{-x} \cos x, \exponential{-x} \sin x}.
  	\end{equation*}
  	Pravá strana zadané rovnice nemá speciální tvar (ani ji na něj nemůžeme převést), takže budeme úlohu řešit variací konstant, tj. partikulární řešení hledáme ve tvaru
  	\begin{equation*}
  		y_p(x) = c_1(x) \exponential{-x} \cos x + c_2(x) \exponential{-x} \sin x,
  	\end{equation*}
  	kde funkce $c_1$, $c_2$ dostaneme vyřešením soustavy
  	\begin{equation}
  		\label{eq:1}
  		\begin{bmatrix}
	  		\exponential{-x} \cos x & \exponential{-x} \sin x
	  		\\
	  		- \exponential{-x} \parens{\cos x + \sin x} & \exponential{-x} \parens{- \sin x + \cos x}
  		\end{bmatrix}
  		\begin{bmatrix}
  			c_1'(x)
  			\\
  			c_2'(x)
  		\end{bmatrix}
  		=
  		\begin{bmatrix}
  			0
  			\\
  			\frac{\exponential{-x}}{\sin x}
  		\end{bmatrix},
  	\end{equation}
  	a následnou integrací nalezených $c_1'$, $c_2'$. Rovnici \eqref{eq:1} můžeme přenásobit $\exponential{x}$ a dostaneme
  	\begin{equation*}
  		\begin{bmatrix}
	  		\cos x & \sin x
	  		\\
	  		- \cos x - \sin x & - \sin x + \cos x
  		\end{bmatrix}
  		\begin{bmatrix}
  			c_1'(x)
  			\\
  			c_2'(x)
  		\end{bmatrix}
  		=
  		\begin{bmatrix}
  			0
  			\\
  			\frac{1}{\sin x}
  		\end{bmatrix}.
  	\end{equation*}
  	Protože
  	\begin{equation*}
  		\det 
  		\begin{bmatrix}
	  		\cos x & \sin x
	  		\\
	  		- \cos x - \sin x & - \sin x + \cos x
  		\end{bmatrix}
  		=
  		- \sin x \cos x + \cos^2 x + \sin x \cos x + \sin^2 x = 1,
  	\end{equation*}
  	podle Cramerova pravidla dostáváme
  	\begin{align*}
  		c_1'(x)
  		&=
  		\det
  		\begin{bmatrix}
	  		0 & \sin x
	  		\\
	  		\frac{1}{\sin x} & - \sin x + \cos x
  		\end{bmatrix}
  		=
  		- 1,
  		&
  		\implies
  		&&
  		c_1(x)
  		&=
  		-x,
  		\\
  		c_2'(x)
  		&=
  		\det
  		\begin{bmatrix}
	  		\cos x & 0
	  		\\
	  		- \cos x - \sin x & \frac{1}{\sin x}
  		\end{bmatrix} 
  		=
  		\frac{\cos x}{\sin x},
  		&
  		\implies
  		&&
  		c_2(x)
  		&=
  		\ln \abs{\sin x}.
  	\end{align*}
  	Obecné řešení definované na intervalech $\parens{k \pi, \parens{k + 1} \pi}$, kde $k \in \Z$, je tedy
  	\begin{equation*}
  		y(x) = C_1 \exponential{-x} \cos x + C_2 \exponential{-x} \sin x - x \exponential{-x} \cos x + \ln \abs{\sin x} \exponential{-x} \sin x,
  	\end{equation*}
  	kde $C_1, C_2 \in \R$.
  \end{solution}
   
\end{questions}

\end{document}
