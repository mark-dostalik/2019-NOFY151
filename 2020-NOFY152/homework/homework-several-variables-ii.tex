\documentclass[answers]{exam}

% encoding, language
\usepackage[T1]{fontenc}
\usepackage[czech]{babel}
\usepackage[utf8]{inputenc}

% geometry
\usepackage[a4paper]{geometry}
\geometry{
  a4paper,
  total={170mm,257mm},
  left=20mm,
  top=20mm}

% mathematics
\usepackage{amsmath}
\usepackage{amssymb}
\usepackage{amsthm}

% text appearence
\usepackage{libertine}
\usepackage{microtype}  % better general appearence of text

% graphics
\usepackage{graphicx}
\graphicspath{{figures/}}
\usepackage{pgfplots}
\usepackage{xcolor}

% colors
\definecolor{LightBlue}{HTML}{42bbed}
\definecolor{LightGray}{HTML}{616a6b}
\definecolor{GraphColor}{HTML}{1d7ad1}

% hypertext
\usepackage{hyperref}
\hypersetup{
	colorlinks=true,
	linkcolor=black,
	urlcolor=LightBlue
}

% bibliography
\usepackage{natbib}

% miscellaneous
\usepackage[parfill]{parskip}
\usepackage{nopageno}
\pagestyle{plain}
\usepackage{titling}
\usepackage{enumitem}

% exam class parameters
\bracketedpoints
\pointpoints{b}{b}
\renewcommand{\solutiontitle}{\noindent\textbf{Řešení: }}

% custom macros
/home/mark/disertation/github/macros.sty

% title
\title{\vspace{-3ex}Matematická analýza II (NOFY152) – DÚ 9}
\author{Věta o implicitní funkci, ODR ve tvaru totálního diferenciálu}
\date{\vspace{-5ex}}

\begin{document}
\maketitle

\begin{questions}	
	\question
		Ukažte, že rovnice
		\begin{equation*}
			z^3 - xz + y = 0,
		\end{equation*}
		na jistém okolí bodu $\parens{x^0, y^0} = \parens{3, -2}$ jednoznačně určuje funkci $z\parens{x, y}$ a spočtěte hodnotu
		\begin{equation*}
			\pd[2]{z}{y}\parens{3, -2}.
		\end{equation*}
		
	\begin{solution}
		Označme si
		\begin{equation*}
			F(x, y, z) = z^3 - xz + y.
		\end{equation*}
		Nejprve najděme bod $z^0$, který spolu s $x^0 = 3$ a $y^0 = -2$ řeší rovnici $F\parens{x^0, y^0, z^0} = 0$. Dosazením za $x^0$, $y^0$ dostáváme rovnici
		\begin{equation*}
			\parens{z^0}^3 -3z^0 - 2 = 0,
		\end{equation*}
		která má dvě řešení $z^{0,1} = -1$ a $z^{0,2} = 2$.
		
		Dále platí
		\begin{equation*}
			\pd{F}{z}\parens{x, y, z} = 3z^2 - x,
		\end{equation*}
		odkud vidíme, že $\pd{F}{z}\parens{x^0, y^0, z^{0,1}} = 0$ a $\pd{F}{z}\parens{x^0, y^0, z^{0,2}} = 9 \neq 0$. Podle věty o implicitní funkci tedy zadaná rovnice na okolí bodu $\parens{x^0, y^0} = \parens{3, -2}$ jednoznačně určuje funkci $z\parens{x,y}$, pro kterou platí $z\parens{3, -2} = z^{0,2} = 2$. Funkce $z$ je navíc na daném okolí dostatečně hladká, neboť $F \in C^{\infty}\parens{\R^3}$.
		
		Na dostatečně malém okolí bodu $\parens{x^0, y^0} = \parens{3, -2}$ nyní derivujme rovnost
		\begin{equation*}
			z^3\parens{x,y} - x z\parens{x, y} + y = 0,
		\end{equation*}
		podle $y$. Dostáváme
		\begin{equation}
			\label{eq:1}
			3 z^2\parens{x,y} \pd{z}{y}\parens{x,y} - x \pd{z}{y}\parens{x,y} + 1 = 0,
		\end{equation}
		odkud po dosazení za $\parens{x, y}$ bod $\parens{x^0, y^0} = \parens{3, -2}$ máme
		\begin{equation}
			\label{eq:2}
			3 \cdot 2^2 \pd{z}{y} \parens{3, -2} - 3 \pd{z}{y} \parens{3, -2} + 1 = 0
			\quad 
			\implies
			\quad
			\pd{z}{y} \parens{3, -2}
			=
			-
			\frac{1}{9}.
		\end{equation}
		Nyní zderivujeme rovnici \eqref{eq:1} opět podle $y$ a dostaneme
		\begin{equation*}
			6 z\parens{x,y} \parens{\pd{z}{y}\parens{x,y}}^2 + \parens{3z^2\parens{x,y} - x} \pd[2]{z}{y}\parens{x,y} = 0.
		\end{equation*}
		Po dosazení za $\parens{x, y}$ bod $\parens{x^0, y^0} = \parens{3, -2}$ pak s využitím \eqref{eq:2} konečně dostáváme
		\begin{equation*}
			6 \cdot 2 \parens{-\frac{1}{9}}^2 + \parens{3 \cdot 2^2 - 3} \pd[2]{z}{y}\parens{3, -2} = 0
			\quad
			\implies
			\quad
			\pd[2]{z}{y}\parens{3, -2} = -\frac{4}{243}.
		\end{equation*}
	\end{solution}
	
  \question 
  	Ukažte, že rovnice 
    \begin{align*}
      x_1\exponential{y_2} + y_1 \ln x_2 - \exponential{} &= 0, \\
      x_1 y_1 + x_2 \exponential{y_2} - (2 + \exponential{} )&=0,
    \end{align*}
    na jistém okolí bodu $\parens{x_1^0, x_2^0} = \parens{1, 1}$ jednoznačně určují funkce $y_1(x_1, x_2)$ a $y_2(x_1, x_2)$ a spočtěte hodnoty
    \begin{equation*}
    	\pd{y_1}{x_1} \parens{1, 1} \quad \text{a} \quad \pd{y_2}{x_1} \parens{1, 1}.
    \end{equation*}
	
  \begin{solution}
		Označme si
    \begin{equation*}
      \vec{x}
      =
      \begin{bmatrix}
        x_1
        \\
        x_2
      \end{bmatrix}
      ,
      \qquad
      \vec{y}
      =
      \begin{bmatrix}
        y_1
        \\
        y_2
      \end{bmatrix},
    \end{equation*}
    a dále
    \begin{equation*}
      \vec{x}_0
      =
      \begin{bmatrix}
        x_1^0
        \\
        x_2^0
      \end{bmatrix}
      =
      \begin{bmatrix}
        1 \\
        1
      \end{bmatrix}
      ,
      \qquad
      \vec{y}_0
      =
      \begin{bmatrix}
        y_1^0
        \\
        y_2^0
      \end{bmatrix},
    \end{equation*}
    a konečně
    \begin{equation*}
      \vec{F}(\vec{x}, \vec{y})
      =
      \begin{bmatrix}
        F_1(\vec{x}, \vec{y}) \\
        F_2(\vec{x}, \vec{y}) \\
      \end{bmatrix}
      =
      \begin{bmatrix}
        x_1\exponential{y_2} + y_1 \ln x_2 - \exponential{} \\
        x_1 y_1 + x_2 \exponential{y_2} - (2 + \exponential{})
      \end{bmatrix}
      .
    \end{equation*}
    Nejprve najdeme bod $\vec{y}_0$, který společně s $\vec{x}_0$ řeší rovnici
    \begin{equation*}
      \vec{F}(\vec{x}_0, \vec{y}_0) = \vec{0},
    \end{equation*}
    aneb chceme, aby platilo
    \begin{equation*}
      \begin{bmatrix}
        x_1^0\exponential{y_2^0} + y_1^0 \ln x_2^0 - \exponential{}  \\
        x_1^0 y_1^0 + x_2^0 \exponential{y_2^0} - (2 + \exponential{})
      \end{bmatrix}
      =
      \begin{bmatrix}
        0 \\
        0
      \end{bmatrix}
      ,
    \end{equation*}
    což po dosazení za $\vec{x}_0$ vede na soustavu rovnic
    \begin{equation*}
      \begin{bmatrix}
        \exponential{y_2^0}  - \exponential{}  \\
        y_1^0 + \exponential{y_2^0} - (2 + \exponential{})
      \end{bmatrix}
      =
      \begin{bmatrix}
        0 \\
        0
      \end{bmatrix}
      ,
    \end{equation*} 
    což dává
    \begin{equation*}
      \vec{y}_0
      =
      \begin{bmatrix}
        y_1^0
        \\
        y_2^0
      \end{bmatrix}
      =
      \begin{bmatrix}
        2 \\
        1
      \end{bmatrix}
      .
    \end{equation*}
    Připomeneme si formální výpočet dle věty o implicitních funkcích. Je-li $\vec{F}(\vec{x}, \vec{y}(\vec{x})) = \vec{0}$, pak
    \begin{equation*}
      \pd{\vec{F}}{\vec{x}} + \pd{\vec{F}}{\vec{y}} \pd{\vec{y}}{\vec{x}} = \vec{0},
    \end{equation*}
    odkud
    \begin{equation*}
      \pd{\vec{y}}{\vec{x}}
      =
      -
      \inverse{
        \left[
          \pd{\vec{F}}{\vec{y}}
        \right]
      }
      \pd{\vec{F}}{\vec{x}}
      ,
    \end{equation*}
    přičemž jsme použili značení
    \begin{equation*}
      \pd{\vec{F}}{\vec{x}}
      =
      \begin{bmatrix}
        \pd{F_1}{x_1} & \pd{F_1}{x_2} \\
        \pd{F_2}{x_1} & \pd{F_2}{x_2}
      \end{bmatrix}
      ,
      \qquad
      \pd{\vec{F}}{\vec{y}}
      =
      \begin{bmatrix}
        \pd{F_1}{y_1} & \pd{F_1}{y_2} \\
        \pd{F_2}{y_1} & \pd{F_2}{y_2}
      \end{bmatrix}
      ,
      \qquad
      \pd{\vec{y}}{\vec{x}}
      =
      \begin{bmatrix}
        \pd{y_1}{x_1} & \pd{y_1}{x_2} \\
        \pd{y_2}{x_1} & \pd{y_2}{x_2}
      \end{bmatrix}
      .
    \end{equation*}
    V našem konkrétním případě dostaneme
    \begin{equation*}
      \pd{\vec{F}}{\vec{x}}
      =
      \begin{bmatrix}
        \exponential{y_2} &  \frac{y_1}{x_2} \\
        y_1 & \exponential{y_2}
      \end{bmatrix}
      ,
      \qquad
      \pd{\vec{F}}{\vec{y}}
      =
      \begin{bmatrix}
        \ln x_2 &  x_1\exponential{y_2} \\
        x_1 &  x_2 \exponential{y_2}
      \end{bmatrix}
      .
    \end{equation*}
    Zajímají nás hodnoty v bodě $\vec{x}_0$. (Připomeňme si, že $\vec{y}(\vec{x}_0) = \vec{y}_0$.) Po dosazení dostaneme
    \begin{equation*}
      \pd{\vec{F}}{\vec{x}} \parens{\vec{x}_0}
      =
      \begin{bmatrix}
        \exponential{} &  2 \\
        2 & \exponential{}
      \end{bmatrix}
      ,
      \qquad
      \pd{\vec{F}}{\vec{y}} \parens{\vec{x}_0}
      =
      \begin{bmatrix}
        0 &  \exponential{} \\
        1 &  \exponential{}
      \end{bmatrix}
      ,
      \qquad
        \inverse{
          \left[
            \pd{\vec{F}}{\vec{y}}
          \right]
        }
      \parens{\vec{x}_0}
      =
      -
      \frac{1}{\exponential{}}
      \begin{bmatrix}
        \exponential{} &  -\exponential{} \\
        -1 &  0
      \end{bmatrix}
      ,
    \end{equation*}
    kde jsme zároveň ověřili, že $\det \parens{\pd{\vec{F}}{\vec{y}} \parens{\vec{x}_0}} = - \exponential{} \neq 0$, a tedy na okolí bodu $\parens{x_1^0, x_2^0} = \parens{1, 1}$ skutečně zadané rovnice jednoznačně určují funkce $y_1(x_1, x_2)$ a $y_2(x_1, x_2)$, které jsou navíc dostatečně hladké.
    
    Požadované derivace najdeme dosazením do vztahu
    \begin{equation*}
      \pd{\vec{y}}{\vec{x}}
      \parens{\vec{x}_0}
      =
      -
      \inverse{
        \left[
          \pd{\vec{F}}{\vec{y}}
        \right]
      }
			\parens{\vec{x}_0}
      \pd{\vec{F}}{\vec{x}}
      \parens{\vec{x}_0}
    \end{equation*}
    což dává
    \begin{equation*}
      \pd{\vec{y}}{\vec{x}}
      \parens{\vec{x}_0}
      =
      \frac{1}{\exponential{}}
      \begin{bmatrix}
        \exponential{} &  -\exponential{} \\
        -1 &  0
      \end{bmatrix}
      \begin{bmatrix}
        \exponential{} &  2 \\
        2 & \exponential{}
      \end{bmatrix}
      =
      \begin{bmatrix}
        \exponential{} -2 &  2 - \exponential{} \\
        -1 & - \frac{2}{\exponential{}}
      \end{bmatrix}
    \end{equation*}
    a proto
    \begin{align*}
      \pd{y_1}{x_1}
      \parens{\vec{x}_0}
      &=
        \exponential{} -2
        ,
      \\
      \pd{y_2}{x_1}
      \parens{\vec{x}_0}
      &=
        -1
        .
    \end{align*}  
	\end{solution}
	
	\question
		Najděte implicitní vztah pro obecné řešení rovnice
		\begin{equation*}
			x^2 y^3 + y + \parens{x^3 y^2 - x} y' = 0.
		\end{equation*}
		\emph{Nápověda:} Převeďte rovnici do rovnice ve tvaru totálního diferenciálu pomocí integračního faktoru $\mu = \mu\parens{xy}$.
	
	\begin{solution}
		Zadanou rovnici můžeme přepsat do tvaru
		\begin{equation*}
			M\parens{x, y} \diff x + N\parens{x, y} \diff y = 0,
		\end{equation*}
		kde
		\begin{align*}
			M\parens{x, y} &= x^2 y^3 + y,
			\\
			N\parens{x, y} &= x^3 y^2 - x.
		\end{align*}
		Nejedná se přímo o rovnici ve tvaru totálního diferenciálu, neboť
		\begin{equation*}
			\pd{N}{x}\parens{x, y} - \pd{M}{y}\parens{x, y}
			=
			3 x^2 y^2  - 1 - 3 x^2 y^2 - 1
			=
			-2
			\neq 
			0.
		\end{equation*}
		Hledejme integrační faktor $\mu = \mu\parens{xy}$, který zajistí platnost
		\begin{equation}
			\label{eq:3}
			\pd{}{x}\parens{\mu\parens{xy}N\parens{x,y}} = \pd{}{y}\parens{\mu\parens{x, y}M\parens{x,y}}.
		\end{equation}
		Rovnice \eqref{eq:3} vede na úlohu
		\begin{equation*}
			\frac{\mu'\parens{xy}}{\mu\parens{xy}}
			=
			\frac{\pd{N}{x}\parens{x,y} - \pd{M}{y}\parens{x,y}}{xM\parens{x,y} - yN\parens{x,y}}
			=
			\frac{-2}{x^3y^3 + xy - x^3y^3 + xy}
			=
			-\frac{1}{xy},
		\end{equation*}
		kde jsme dosadili za $M$ a $N$. Dostáváme tedy lineární diferenciální rovnici 1. řádu
		\begin{equation*}
			\frac{\mu'(z)}{\mu(z)} = -\frac{1}{z},
		\end{equation*}
		jejímž řešením je na intervalech $\parens{-\infty, 0}$ a $\parens{0, +\infty}$ funkce $\mu(z) = \frac{c}{\abs{z}}$, kde $c \in \R \setminus \set{0}$. Volme tedy nejjednoduší integrační faktor $\mu(xy) = \frac{1}{xy}$. (Multiplikativní konstanty nehrají roli.) V dalším tedy řešíme zadanou rovnici na čtyřech podoblastech $\R^2$ (mimo souřadné osy).
		
		Přenásobení zadané rovnice nalezeným integračním faktorem vede na rovnici ve tvaru totálního diferenciálu
		\begin{equation*}
			\parens{xy^2 + \frac{1}{x}} \diff x + \parens{x^2y-\frac{1}{y}} \diff y = 0,
		\end{equation*}
		odkud dostáváme pro její potenciál $U$
		\begin{equation*}
			U\parens{x,y}
			=
			\int
				\parens{xy^2 + \frac{1}{x}}
			\diff x
			=
			\frac{1}{2}x^2y^2 + \ln \abs{x} + \varphi(y),
		\end{equation*}
		a
		\begin{equation*}
			x^2y - \frac{1}{y}
			=
			\pd{U}{y}\parens{x,y}
			=
			x^2 y + \varphi'(y)
			\quad
			\implies
			\quad
			\varphi'(y) = - \frac{1}{y}
			\quad
			\implies
			\quad
			\varphi(y) = - \ln \abs{y} - C,
		\end{equation*}
		kde $C \in \R$. Celkově má tedy zadaná rovnice řešení dané implicitně předpisem
		\begin{equation*}
			\frac{1}{2} x^2 y^2 + \ln \abs{\frac{x}{y}} = C,
		\end{equation*}
		kde konstantu $C$ určíme z počáteční podmínky.
	\end{solution}
  
\end{questions}

\end{document}
