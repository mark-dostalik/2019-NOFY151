\documentclass[answers]{exam}

% encoding, language
\usepackage[T1]{fontenc}
\usepackage[czech]{babel}
\usepackage[utf8]{inputenc}

% geometry
\usepackage[a4paper]{geometry}
\geometry{
  a4paper,
  total={170mm,257mm},
  left=20mm,
  top=20mm}

% mathematics
\usepackage{amsmath}
\usepackage{amssymb}
\usepackage{amsthm}

% text appearence
\usepackage{libertine}
\usepackage{microtype}  % better general appearence of text

% graphics
\usepackage{graphicx}
\graphicspath{{figures/}}
\usepackage{pgfplots}
\usepackage{xcolor}

% colors
\definecolor{LightBlue}{HTML}{42bbed}
\definecolor{LightGray}{HTML}{616a6b}
\definecolor{GraphColor}{HTML}{1d7ad1}

% hypertext
\usepackage{hyperref}
\hypersetup{
	colorlinks=true,
	linkcolor=black,
	urlcolor=LightBlue
}

% bibliography
\usepackage{natbib}

% miscellaneous
\usepackage[parfill]{parskip}
\usepackage{nopageno}
\pagestyle{plain}
\usepackage{titling}
\usepackage{enumitem}

% exam class parameters
\bracketedpoints
\pointpoints{b}{b}
\renewcommand{\solutiontitle}{\noindent\textbf{Řešení: }}

% custom macros
% correct spacing for \left \right commands
\let\originalleft\left
\let\originalright\right
\renewcommand{\left}{\mathopen{}\mathclose\bgroup\originalleft}
\renewcommand{\right}{\aftergroup\egroup\originalright}

% note/suggestion boxes
\newcommand*{\note}[1]{\smallskip\noindent\colorbox{SkyBlue}{
  \begin{minipage}{\linewidth}
  \textbf{Note }#1
  \end{minipage}}\smallskip
}
\newcommand*{\suggestion}[1]{\smallskip\noindent\colorbox{Dandelion}{
  \begin{minipage}{\linewidth}
  \textbf{Suggestion }#1
  \end{minipage}}\smallskip
}

% calculus (miscellaneous)
\newcommand*{\diff}{\mathrm{d}}
\newcommand*{\Diff}{\mathrm{D}}
\newcommand*{\loc}{\mathrm{loc}}
\newcommand*{\iunit}{\mathrm{i}}
\newcommand*{\inverse}[1]{#1^{-1}}
\newcommand*{\realpart}[1]{\mathrm{Re} \parens{#1}}
\newcommand*{\imagpart}[1]{\mathrm{Im} \parens{#1}}

% sets
\newcommand*{\set}[2]{\braces{\, #1 \mid #2 \,}}
\newcommand*{\N}{\ensuremath{{\mathbb{N}}}}
\newcommand*{\R}{\ensuremath{{\mathbb{R}}}}

% definition sign
\newcommand*{\defeq}{\mathrel{\overset{\makebox[0pt]{\mbox{\normalfont\tiny\sffamily def}}}{=}}}

% parens, brackets, braces, angles
\newcommand*{\parens}[1]{\left( #1 \right)}
\newcommand*{\brackets}[1]{\left[ #1 \right]}
\newcommand*{\braces}[1]{\left\{ #1 \right\}}
\newcommand*{\angles}[1]{\langle #1 \rangle}

% norms
\newcommand*{\abs}[2][]{\ensuremath{\left|#2\right|_{#1}}}
\newcommand*{\norm}[2][]{\ensuremath{\left\|#2\right\|_{#1}}}

% inner products
\newcommand*{\inner}[3][parens]{%
  \IfEqCase{#1}{%
    {parens}{\parens{#2, #3}}%
    {angles}{\angles{#2, #3}}%
  }[\PackageError{inner}{Undefined option to inner: #1}{}]%
}

% function spaces
\newcommand*{\lebs}[2]{L^{#1} \parens{#2}}
\newcommand*{\leblocs}[2]{L^{#1}_{\loc} \parens{#2}}


% title
\title{\vspace{-3ex}Matematická analýza II (NOFY152) – DÚ 4}
\author{Taylorův polynom, určitý integrál, ODR se separovanými proměnnými}
\date{\vspace{-5ex}}

\begin{document}
\maketitle

\begin{questions}
  \question Pomocí Taylorova polynomu spočtěte
	\begin{enumerate}[label=(\roman*)]
		\item
		  \begin{equation*}
		    \lim_{x \to 0}
		    \frac{\cos x - 1 + \frac{1}{2} x \sin x}{\ln^4 \parens{1 + x}},
		  \end{equation*}
		\item
		  \begin{equation*}
		    \lim_{x \to 0}
		    \frac{1 - \parens{\cos x}^{\sin x}}{x^3}.
		  \end{equation*}
	\end{enumerate}

  Při výpočtu nepoužívejte l'H\^{o}spitalovo pravidlo ani znalost základních limit.
  
  \begin{solution}
		\begin{enumerate}[label=(\roman*)]
			\item
		  	Uvažujme následující Taylorovy rozvoje v počátku (tj. pro $x \to 0$)
		  	\begin{align*}
		  		\sin x 
		  		&= 
		  		x - \frac{x^3}{3!} + o(x^4),
		  		\\
		  		\cos x
		  		&=
		  		1 - \frac{x^2}{2} + \frac{x^4}{4!} + o(x^5),
		  		\\
		  		\ln (1 + x)
		  		&=
		  		x + o(x).	
		  	\end{align*}
		  	Po dosazení dostáváme
		  	\begin{align*}
		    \lim_{x \to 0}
		    \frac{\cos x - 1 + \frac{1}{2} x \sin x}{\ln^4 \parens{1 + x}}
		    &=
		    \lim_{x \to 0}
		    \frac{1 - \frac{x^2}{2} + \frac{x^4}{4!} + o(x^5) - 1 + \frac{1}{2}x\parens{x - \frac{x^3}{3!} + o(x^4)}}{\parens{x + o(x)}^4}
		    \\
		    &=
		    \lim_{x \to 0}
		    \frac{\parens{\frac{1}{4!} - \frac{1}{2 \cdot 3!}} x^4 + o(x^5)}{x^4 + o(x^4)} 
		    =
		    \lim_{x \to 0}
		    \frac{-\frac{1}{24}x^4}{x^4 + o(x^4)}
		    +
		    \lim_{x \to 0}
		    \frac{o(x^4)}{x^4 + o(x^4)}
		    \\
		    &=
		    \lim_{x \to 0}
		    \frac{-\frac{1}{24}}{1 + \frac{o(x^4)}{x^4}}
		    +
		    \lim_{x \to 0}
		    \frac{\frac{o(x^4)}{x^4}}{1 + \frac{o(x^4)}{x^4}}   
		    =
		    -\frac{1}{24},
		  	\end{align*}  
		  	kde jsme při výpočtu využili aritmetiku limit.
			\item
		  	Užitím následujících Taylorových rozvojů v počátku (tj. pro $x \to 0$)
		  	\begin{align*}
		  		\sin x 
		  		&= 
		  		x - \frac{x^3}{3!} + o(x^4),
		  		\\
		  		\cos x
		  		&=
		  		1 - \frac{x^2}{2} + o(x^3),
		  		\\
		  		\ln (1 + x)
		  		&=
		  		x + o(x),
		  		\\
		  		\exponential{x}
		  		&=
		  		1 + x + o(x),
		  	\end{align*}
		  	postupně dostáváme
		  	\begin{align*}
		  		\parens{\cos x}^{\sin x}
		  		&=
		  		\exponential{\sin x \ln \parens{\cos x}}
		  		=
		  		\exponential{\parens{x - \frac{x^3}{3!} + o(x^4)} \ln \parens{1 - \frac{x^2}{2} + o(x^3)}}
		  		=
		  		\exponential{\parens{x - \frac{x^3}{3!} + o(x^4)} \parens{-\frac{x^2}{2} + o(x^2)}}
		  		\\
		  		&=
		  		\exponential{-\frac{x^3}{2} + o(x^3)}
		  		=
		  		1 - \frac{x^3}{2} + o(x^3).
		  	\end{align*}
		  	Po dosazení tedy máme
		  	\begin{align*}
		    \lim_{x \to 0}
		    \frac{1 - \parens{\cos x}^{\sin x}}{x^3}
		    =
		    \lim_{x \to 0}
		    \frac{1 - 1 + \frac{x^3}{2} + o(x^3)}{x^3}
		    =
		    \frac{1}{2}.
		  	\end{align*}
		\end{enumerate}

  \end{solution}

  \question Spočtěte objem tělesa, které vznikne rotací oblasti 
  \begin{equation*}
  	\set[
  		0 \le x \le 1, \, 0 \le y \le \frac{\ln \frac{1}{x}}{\parens{1 + x^2}^2}
  	]{
  		\parens{x, y} \in \R^2
  	}
  \end{equation*}
  kolem osy $y$.
  
  \begin{solution}
  	Pro hledaný objem $V$ platí 
  	\begin{equation*}
  		V 
  		= 
  		2 \pi \int_{0}^{1} \frac{x \ln \frac{1}{x}}{\parens{1 + x^2}^2} \, \diff x.
  	\end{equation*}
  	Označme si primitivní funkci integrandu
  	\begin{equation*}
  		F(x)
  		\defeq
  		\int \frac{x \ln \frac{1}{x}}{\parens{1 + x^2}^2} \, \diff x,
  	\end{equation*}
  	a metodou per partes počítejme (integrační konstanta není důležitá, tak ji nepíšeme)
  	\begin{align*}
  		F(x)
  		&=
      \left| 
        \begin{aligned}
          u &= \ln \frac{1}{x} & v &= \frac{-1}{2\parens{1 + x^2}}
          \\
          u' &= -\frac{1}{x} & v' &= \frac{x}{\parens{1 + x^2}^2}
        \end{aligned}
      \right|
      =
      \frac{\ln x}{2 \parens{1 + x^2}}
      -
      \frac{1}{2} \int \frac{1}{x \parens{1 + x^2}} \, \diff x
      \\
      &=
      \frac{\ln x}{2 \parens{1 + x^2}}
      -
      \frac{1}{2}
      \int
      	\parens{\frac{1}{x} - \frac{x}{1 + x^2}}
      \diff x
      =
      \frac{\ln x}{2 \parens{1 + x^2}}
      -
      \frac{1}{2}
      \parens{\ln x - \frac{1}{2} \ln \parens{1 + x^2}}
      \\
      &=
      \frac{1}{4} \ln \parens{1 + x^2}
      -
      \frac{x^2 \ln x}{2 \parens{1 + x^2}}.
  	\end{align*}
  	Všimněte si, že v tomto příkladě nelze využít metodu per partes pro Newtonův integrál, neboť pro první výraz za druhým rovnítkem výše platí
  	\begin{equation*}
  		\lim_{x \to 0+} 
  		\frac{\ln x}{2 \parens{1 + x^2}} = -\infty.
  	\end{equation*}
  	(Podívejte se na přesné znění příslušné věty.)
  	
  	Limity primitivní funkce $F$ v krajních bodech vychází následovně
  	\begin{align*}
  		\lim_{x \to 1-} F(x)
  		&=
  		\lim_{x \to 1-}
  		\parens{
	  		\frac{1}{4} \ln \parens{1 + x^2}
	  		-
	  		\frac{x^2 \ln x}{2 \parens{1 + x^2}}
  		}
  		=
  		\frac{1}{4} \ln 2,
  		\\
  		\lim_{x \to 0+} F(x)
  		&=
  		\lim_{x \to 0+}
  		\parens{
	  		\frac{1}{4} \ln \parens{1 + x^2}
	  		-
	  		\frac{x^2 \ln x}{2 \parens{1 + x^2}}
  		}
  		=
  		\frac{1}{2} \lim_{x \to 0+}
  		\frac{\ln x}{1 + \frac{1}{x^2}}
  		\texteq{l'H}
  		\frac{1}{2} \lim_{x \to 0+}
  		\frac{\frac{1}{x}}{-\frac{2}{x^3}}  
  		=
  		0,		
  	\end{align*}
    kde jsme v předposlední rovnosti využili l'Hôspitalova pravidla pro výpočet limity typu ``$\frac{\textrm{něco}}{\infty}$''.  
    	
  	Nakonec tedy dostáváme
  	\begin{equation*}
  		V 
  		= 
  		2 \pi \parens{ \lim_{x \to 1-} F(x) - \lim_{x \to 0+} F(x)}
  		=
  		2 \pi \parens{\frac{1}{4} \ln 2 - 0}
  		=
  		\frac{\pi}{2} \ln 2.
  	\end{equation*}
  \end{solution}

  \question Pro diferenciální rovnici
  \begin{equation*}
    \dd{y}{x}
    =
    2 \exponential{-x} \sqrt{1 - y},
  \end{equation*}
  nalezněte
	\begin{enumerate}[label=(\roman*)]
		\item všechna maximální řešení,
		\item všechna maximální řešení splňující $y(0) = 1$.
	\end{enumerate}
  
  \begin{solution}
	\begin{enumerate}[label=(\roman*)]
		\item
	  	Diferenciální rovnice je ve tvaru
	  	\begin{equation*}
	    \dd{y}{x}
	    =
	    f(x) g(y), 	
	  	\end{equation*}
	  	kde $f(x) = \exponential{-x}$, $g(y) = 2 \sqrt{1 - y}$ a můžeme ji tedy řešit metodou separace proměnných. Rovnice má smysl pro $x$ z intervalu $I = \R$. Na celé reálné ose máme stacionární řešení $y \equiv 1$. Zbylá řešení $y$ hledáme z intervalu $J = \parens{-\infty, 1}$, kde je $g$ spojitá a nenulová.
	  	
	  	Spočtěme primitivní funkce
	  	\begin{align*}
	  		F(x)
	  		&\defeq
	  		\int f(x) \, \diff x
	  		=
	  		\int \exponential{-x} \, \diff x
	  		=
	  		- \exponential{-x} - c, \quad c \in \R,
	  		\\
	  		G(y)
	  		&\defeq
	  		\int \frac{1}{g(y)} \, \diff y
	  		=
	  		\int \frac{1}{2 \sqrt{1 - y}} \, \diff y
	  		=
	      - \sqrt{1 - y}.
	  	\end{align*}
	  	Z jejich rovnosti
	  	\begin{equation*}
	  		\sqrt{1 - y}
	  		=
	  		\exponential{-x} + c,
	  	\end{equation*}
	  	a nezápornosti levé strany potom dostáváme
	  	\begin{itemize}
	  		\item $c \ge 0 \implies x \in \R$,
	  		\item $c < 0 \implies x \in \parens{-\infty, -\ln\parens{-c}}$.
	  	\end{itemize}
	  	
	  	Všechna maximální řešení tedy jsou
	  	\begin{align*}
	  		y(x) &= 1, \quad x \in \R
	  		\\
	  		y(x) &= 1 - \parens{\exponential{-x} + c}^2, \quad x \in \R, \quad c \ge 0,
	  		\\
	  		y(x) &=
	  		\begin{cases}
	  			1 - \parens{\exponential{-x} + c}^2, \quad &x \in \parens{-\infty, -\ln\parens{-c}},
	  			\\
	  			1, \quad &x \in \left[-\ln\parens{-c}, +\infty \right),
	  		\end{cases}
	  		\quad c < 0,
	  	\end{align*}
	  	kde jsme v posledním případě navíc slepili nalezené řešení se stacionárním.
	  \item 
	  	Všimněme si nejdřív, že stacionární řešení $y \equiv 1$ splňuje počáteční podmínku $y(0) = 1$. Dále najděme hodnotu konstanty $c$, pro kterou bude i netriviální řešení splňovat počáteční podmínku
	  	\begin{equation*}
	  		1 = y(0) = 1 - \parens{1 +c}^2 \implies c = -1.
	  	\end{equation*}
	  	Díky napojování na stacionární řešení pak dostáváme, že i všechna netriviální řešení s $c \le -1$ splňují počáteční podmínku.
	  	
	  	Maximální řešení splňující počáteční podmínku $y(0) = 1$ tedy jsou
	  	\begin{align*}
	  		y(x) &= 1, \quad x \in \R
	  		\\
	  		y(x) &=
	  		\begin{cases}
	  			1 - \parens{\exponential{-x} + c}^2, \quad &x \in \parens{-\infty, -\ln\parens{-c}},
	  			\\
	  			1, \quad &x \in \left[-\ln\parens{-c}, +\infty \right),
	  		\end{cases}
	  		\quad c \le -1.
	  	\end{align*}

	\end{enumerate}  
	
	\begin{center}
     \begin{tikzpicture}
       \begin{axis}[
  	      width=0.85\textwidth,
  	      height=0.635\textwidth,
        	samples=500,
        	domain=-3:3,
        	restrict y to domain =-1:1.25,
         xlabel=$x$, 
         xmin=-1,
         xmax=1,
         xtick pos=bottom,
         xtick={-1,-0.693147, -0.405465, 0, 0.693147, 1},
         xticklabels={$-1$, $-\ln2$, $-\ln\frac{3}{2}$, $0$, $-\ln\frac{1}{2}$, $1$},
         ylabel style={rotate=-90, xshift=7pt},
         ylabel=$y(x)$,
         ymin=-1,
         ymax=1.25,
         ytick pos=left,
         ytick={-1, 0, 1},
         yticklabels={$-1$,$0$,$1$},]
          \addplot[blue, samples=1000, smooth, color=GraphColor, line width=0.8pt, domain=-3:3] coordinates {(-4,1) (3,1)};	        
         \addplot[
           blue, samples=1000, smooth, color=GraphColor, line width=0.8pt, domain=-3:3
         ]{1-(exp(-x)+0.5)^2};	 
         \addplot[
           blue, samples=1000, smooth, color=GraphColor, line width=0.8pt, domain=-3:3
         ]{1-(exp(-x))^2};	
         \addplot[
           blue, samples=1000, smooth, color=GraphColor, line width=0.8pt, domain=-3:0.693147
         ]{1-(exp(-x)-0.5)^2};	                                       
         \addplot[
           blue, samples=1000, smooth, color=GraphColor, line width=0.8pt, domain=-4:-0.0001
         ]{1-(exp(-x)-1)^2};
         \addplot[
           blue, samples=1000, smooth, color=GraphColor, line width=0.8pt, domain=-4:-0.405465
         ]{1-(exp(-x)-1.5)^2};	          
         \addplot[
           blue, samples=1000, smooth, color=GraphColor, line width=0.8pt, domain=-4:-0.693147
         ]{1-(exp(-x)-2)^2};
         \addplot[
           only marks, mark size=1.5pt, color=GraphColor, samples at = {0.693147}
         ]{1-(exp(-x)-0.5)^2};	          
         \addplot[
           only marks, mark size=1.5pt, color=GraphColor, samples at = {0}
         ]{1-(exp(-x)-1)^2};
         \addplot[
           only marks, mark size=1.5pt, color=GraphColor, samples at = {-0.405465}
         ]{1-(exp(-x)-1.5)^2};
         \addplot[
           only marks, mark size=1.5pt, color=GraphColor, samples at = {-0.693147}
         ]{1-(exp(-x)-2)^2};	
         \addplot[dotted] coordinates {(-0.693147,-1) (-0.693147,1)};
         \addplot[dotted] coordinates {(-0.405465,-1) (-0.405465,1)};
         \addplot[dotted] coordinates {(0,-1) (0,1)};
         \addplot[dotted] coordinates {(0.693147,-1) (0.693147,1)};
         \node[] at (axis cs: 0.55,-0.4) {$c=\frac{1}{2}$}; 
         \node[] at (axis cs: 0.08,-0.08) {$c=0$}; 
         \node[] at (axis cs: -0.22,0.065) {$c=-\frac{1}{2}$};
         \node[] at (axis cs: -0.47,0.3) {$c=-1$}; 
         \node[] at (axis cs: -0.63,0.6) {$c=-\frac{3}{2}$};
         \node[] at (axis cs: -0.79,0.8) {$c=-2$};   	             
       \end{axis}
     \end{tikzpicture}		
	\end{center}

  \end{solution}
   
\end{questions}

\end{document}
